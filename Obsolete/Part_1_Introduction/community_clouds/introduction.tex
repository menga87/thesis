
% Section: INTRODUCTION
%\section{Introduction}
%\label{sec:introduction}

%Cloud computing infrastructures have emerged as a cost-effective, elastic and scalable way to build and support Internet applications, but issues like privacy, security, control over data and applications, performance, reliability, availability and access to specific cloud services have led to different cloud deployment models.
%Among these deployment models, the public cloud offers services of generic interest over the Internet, available to anybody who signs in with its credentials. 
%On the other side, the private cloud model aims to provide cloud services to only a specific user group, such as a company, and the cloud infrastructure is isolated by firewalls avoiding public access. 
%Finally, when a private cloud is combined with the public cloud, for instance some functionality of the cloud is provided by the public cloud and some remains in the private cloud, then this cloud model is called hybrid cloud. 
%The community cloud bridges in different aspects the gap between the public cloud, the general purpose cloud available to all users, and the private cloud, available to only one cloud user with user-specific services. 
%The concept of community cloud computing has been described in its generic form in~\cite{Mell2011} as a cloud deployment model in which a cloud infrastructure is built and provisioned for an exclusive use by a specific community of consumers with shared concerns and interests, owned and managed by the community or by a third party or a combination of both.
%The community cloud model assumes that cloud users can be classified into communities, where each community of users has specific needs. Hence for such communities a specific cloud, the community cloud, addresses through its particular properties the cloud requirements of such a group of users.
%
%We elaborate the characterization of community clouds in the rest of this chapter as follows.
%In section~\ref{sec:background} we explain the concept of community cloud computing in detail. 
%In section~\ref{sec:audience} we give examples of cloud user communities for which the community cloud model could find application. 
%In section~\ref{sec:requirements} we derive the requirements on community clouds, in relation to user communities. 
%We present in section~\ref{sec:applications} the potential and advantages of community clouds, and in section~\ref{sec:achievements} how existing community cloud solution fulfil the challenges for their communities. 
%Finally in section~\ref{sec:conclusion} we conclude our study on community clouds and outline some future scenarios based on current trends.


%%%%%%% Copying this from CloudNet'15 paper

% <copied> % this text Wiley Encyclopedia Chapter
Cloud computing has emerged as a cost-effective, elastic and scalable way to build and support Internet applications, 
but issues like privacy, security, control over data and applications, performance, reliability, availability and access to specific cloud services have led to different cloud deployment models.
Among these deployment models, the public cloud offers services of generic interest over the Internet, available to anybody who signs in with its credentials. 
On the other side, the private cloud model aims to provide cloud services to only a specific user group, 
such as a company, and the cloud infrastructure is isolated by firewalls avoiding public access. 
Finally, when a private cloud is combined with the public cloud, for instance some functionality of the cloud is provided by the public cloud and some remains in the private cloud, this is referred to as hybrid cloud.

The community cloud bridges in different aspects the gap between the public cloud, the general purpose cloud available to all users, %~\cite{Khan2015CommunityClouds}
and the private cloud, available to only one cloud user with user-specific services. 
The concept of community cloud computing has been introduced in its generic form before, e.g.~\cite{Mell2011}, as a cloud deployment model in which a cloud infrastructure is built and provisioned for an exclusive use by a specific community of consumers with shared concerns and interests, owned and managed by the community or by a third party or a combination of both.
The community cloud model assumes that cloud users can be classified into communities, where each community of users has specific needs. 
Hence for such communities a specific cloud, the community cloud, addresses through its particular properties the cloud requirements of such a group of users.

% <copied> % this text Wiley Encyclopedia Chapter
%Community clouds are implemented using different designs depending upon the requirements. 
%One common approach is that a public cloud provider sets up separate infrastructure and develops services specifically for a community to provide a vertically integrated solution for that market. 
%Similarly, a third party service provider can focus on a particular community and only specialise in building tailor-made solutions for that community.
%Another option is that community members who already have expertise in cloud infrastructures come together to federate their private clouds and collectively provision cloud services to the community. %~\cite{Buyya2010InterCloud}
%Another radical model involves building community cloud services using resources contributed by individual users by either solely relying on user machines or using them to augment existing cloud infrastructures~\cite{Marinos2009}.

Just as cloud computing can involve variety of definitions depending upon how the services are provisioned and consumed, 
community cloud computing can have different interpretations depending upon the specific requirements and characteristics of the community, 
and how the infrastructures are deployed and services are provisioned.
In this thesis, our focus is on the collaborative model of provisioning community cloud services based on volunteer computing paradigm as laid out in~\cite{Marinos2009}.
We study the existing work and analyse the current trends and the future research directions of this collaboration based community cloud computing model.
We then explore this in the context of community networks which are a successful case of collaboratively built communication infrastructure, and we see community cloud services as the logical next step in the evolution of community networks.

%\subsection{What is community cloud computing?}
%Community cloud computing has been defined~\cite{Mell2011} as infrastructures set up to support causes of a community of users. 
%This can include services provisioned by a single commercial public cloud provider, like 	Amazon Web Services\footnote{\url{http://aws.amazon.com}}, Google Cloud Platform\footnote{\url{http://cloud.google.com}} and Windows Azure\footnote{\url{http://windowsazure.com}}, and consumed by the community.
%But this also applies to the scenario where multiple third-party or community's own cloud providers provide the services. Our focus in this paper is when community provides and consumes cloud services using resources of its own. Having more than one provider is inevitable so this idea is closely tied to federated clouds, also know as InterCloud and Multi-Cloud in the literature. So for our purpose, we define community cloud computing as: \emph{Cloud infrastructures provisioned collectively by the community serving the common interests of the community.}

%We elaborate our study in the rest of the paper as following.
%We study the different initiatives in community cloud computing in section~\ref{sec:initiatives}, and focus on community network clouds in section~\ref{sec:cn-clouds}. 
%We discuss future directions and conclude our study in section~\ref{sec:conclusion}.
