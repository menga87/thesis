%
%
%\subsection{Potential of Community Clouds}
%\label{sec__advantages_community_clouds}
%
%This section states the potential of community clouds, highlighting their key advantages.
%
%\subsubsection{Security, Control and Privacy}
%In some cases, the security and privacy of data is so important that giving access and control of this information to a public cloud provider is not feasible.
%Private clouds are also not an option because of the huge investment needed to set up and maintain such infrastructure.
%Community clouds address both of these concerns provided that the members of the community have existing trust relationships among themselves and require similar applications.
%%DLA Piper's Data 'Protection Laws of the World Handbook' gives a comprehensive overview of the regulations dealong with data protection: http://dlapiperdataprotection.com/
%
%\subsubsection{Elasticity, Resilience and Robustness}
%Community clouds bring co-operating entities together that can pool their infrastructures. 
%In contrast to private clouds, this provides redundancy and robustness, and result in a more resilient system unlike public cloud providers which can act as a point of failure as evident from the outages, even though rare, at many public providers in recent years. 
%In the situation of peaks in demand and outages at the location of one member, resources from other members can help in alleviating these problems.
%
%\subsubsection{Avoiding Lock-In}
%Consuming services from a public cloud suffers from vendor lock-in problem.
%In the absence of standardisation in clouds, moving data and applications from one cloud vendor to another is not feasible.
%Community cloud allows to consume resources either from a variety of vendors or from infrastructure tailored to the community, and so provides a protection against vendor lock-in.
%
%\subsubsection{Cost Effectiveness}
%Private and hybrid clouds require huge capital investment for building and maintaining the infrastructure in-house.
%Community clouds allow to distribute this cost among the members of the community.
%
%\subsubsection{Enhanced Requirement Satisfaction}
%In some cases, the public clouds may not be able to meet the performance and functionality demands of the community.
%For example, the applications may be time critical and need better network speed than a public cloud service provider can commit.
%
