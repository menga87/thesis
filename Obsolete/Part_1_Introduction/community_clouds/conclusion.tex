%
%
%% Section: CONCLUSION
%\section{Conclusion}
%\label{sec:conclusion}
%
%%[Conclusion: Summarize your topic and briefly touch on future prospects (if applicable). You might sketch out future scenarios or issues likely to arise from current trends related to your 
%%topic. Concluding remarks should follow coherently from information presented earlier in the body of the article.]
%
%A community cloud is a cloud deployment model in which a cloud infrastructure is built and provisioned for an exclusive use by a specific community of consumers with shared concerns and interests, owned and managed by the community or by a third party or a combination of both.
%The wide offer of commercial cloud solutions has led to a widespread adoption of cloud usage by all kind of stakeholders. 
%It is a natural evolution from this growing number of cloud users that within these users, certain clusters or communities of users arise, where each community is characterised by shared interests and common concerns.
%Community clouds have become a reality today with community cloud solutions deployed in many important economic sectors and for different stakeholders, like manufacturing industry, telecommunication providers, financial service providers, health services, government agencies and education. 
%Among the advantages of community clouds is that the specific cloud user requirements are satisfied and in a more cost-effective way. 
%High security standards and high cloud performance can be achieved with community clouds at a cost that is shared among the community.   
%The opportunity of community cloud lies in being able to offer optimised cloud solutions to specific user communities. 
%An important condition however is that the requirements of the community are shared, in order to be able to determine a specific cloud solution, which is worth the effort developing a community cloud. 
%
