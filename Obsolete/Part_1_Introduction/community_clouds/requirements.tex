%
%
%%% Section: REQUIREMENTS
%\section{Requirements and Challenges}
%\label{sec:requirements}
%
%In the following, requirements on the community clouds emphasised by different user communities are stated. 
%These requirements provide the foundation for the design of the community cloud system, and need to be satisfied for it to be deployed and adopted successfully by the community.
%
%\subsection{Security}
%There are many security challenges that need to be addressed for ensuring users' trust in the system. 
%With multiple independent cloud providers from the community, security becomes even more important in a community cloud. 
%The data and applications running on different cloud systems has to be protected from unauthorised access, damage and tampering.
%With the rise in sophisticated security threats and increase in data breaches, this is one of the most critical factor impacting any community cloud system.
% 
%
%\subsection{Autonomy}
%Community cloud systems may be formed based on individual cloud systems that are set up and managed independently by different owners. 
%%We cannot assume or require prior coordination or even trust between different cloud owners.
%%This means that each cloud owner can take decisions about his or her cloud setup without negotiating with other partners beforehand. 
%The main requirement for a cloud owner for participating in such a community cloud is that the local cloud setup should adhere to the common API provided by the community cloud. 
%In addition, the cloud owner should contribute some set of mutually agreed resources to the community.
%
%
%\subsection{Self-Management}
%Depending on the type of community, self-management capabilities may be an important requirement on the community cloud system. 
%%Community cloud nodes may join and leave the community cloud at any time.
%Community cloud should manage itself and continue providing services without disruption, even when part of the infrastructure run into issues. 
%Self-management should also help in the coordination between different cloud owners that become part of a federated community cloud.
%
%
%\subsection{Utility}
%For the acceptance of the community cloud, it should provide applications that are valuable for the community. 
%Usage strengthens the value of the community cloud, motivating its maintenance and update.
%
%
%%\subsection{Ease of Use}
%%Most of the users of the community cloud will not be proficient in cloud technologies, so setting up nodes for deployment and managing cloud software should be simple and straightforward.
%%Similarly, managing and updating the cloud software on the nodes should be as automatic as possible.
%
%
%\subsection{Incentives for Contribution}
%Some flavours of community clouds may be build upon collective efforts. 
%Such a community cloud builds on the contribution of the volunteers in terms of computing, storage, network resources, time and knowledge.
%For these community clouds to be sustainable, incentive mechanisms are needed to encourage users to actively contribute towards the system.
%
%
%\subsection{Support for Heterogeneity}
%The hardware and software used by members in a community cloud can have quite varying characteristics, which can result from the usage of different hardware, operating systems, cloud platforms or application software. 
%The community cloud system should handle this heterogeneity seamlessly.
%%There could be powerful machines with good network connectivity and abundant storage space.
%%On the other end, there are less powerful machines with limited CPU, RAM, disk space and bandwidth.
% 
%
%
%\subsection{Standard API}
%The cloud system should make it straightforward for the application programmers to design their applications in a transparent manner for the underlying heterogeneous cloud infrastructure.
%The API should provide the appearance of a middleware that obviates the need to customise the applications specific to each cloud architecture.
%This is essential for community clouds when these result from the federation of many independently managed clouds.
%Each such cloud may be using a different cloud management platform that may provide a different set of API.
%Providing a standard API for the community cloud ensures that applications written for one community cloud can also be deployed for another community cloud in the future and as they are integrated into the community cloud they can be easily deployed on new cloud architectures. 
%
%
%\subsection{QoS and SLA Guarantees}
%The community cloud system needs mechanisms for ensuring quality of service (QoS) and enforcing service level agreements (SLA).
%With business processes increasingly reliant on the cloud services and applications, businesses need strong guarantees before transitioning to a community cloud system.
