
% Section: INTRODUCTION
%\subsubsection{Community Networks}
%\section{Clouds in Community Networks}
\label{sec__community_network_clouds}

%\subsection{Extending the Sharing of Bandwidth to the Sharing of Computing Resources}

Community networking~\cite{Braem2013} is a shared communication infrastructure in which citizens build and own open communication networks, using mostly wireless but also optical fibre links. 
%Most of these community networks are based on Wi-Fi technology such as ad-hoc networks or IEEE 802.11a/b/g/n access points in the first hop and long-distance point-to-point Wi-Fi links for the trunk network.
%There are several large community networks in Europe, having from 500 to 20,000 nodes, such as 
%	Athens Wireless Metropolitan Network (AWMN), 
%	Freifunk,
%	FunkFeuer,
%	Guifi.net,
%	Ninux,
%and many others worldwide. 
%
Community networks are already based on the principle of sharing, though only of bandwidth, but the social aspects and community nature makes it easier to extend this sharing to other computing resources~\cite{Khan2015Incentive}.
%~\cite{Khan2013Clouds, Jimenez2013, Jimenez2014, Selimi2014TowardsApplication, Selimi2014Distributed, Selimi2014Experiences, Liu2014, Selimi2014Cloud}.
%However, challenges peculiar to such networks are limited, unstable and variable network bandwidth and high churn of nodes.
The strong sense of community and technical knowledge of participants of such networks are some strong points which are conducive to building cloud applications tailored to local needs built on infrastructure provided by the community members.
% \cite{Khan2014Sparks,Khan2014Architecture,Khan2014Macroeconomic}.
For example, Guifi.net community network
%Guifi.net, one of the largest community networks in the world with more than 2800 nodes, 
has deployed community edge cloud using their Debian-based Cloudy distribution~\cite{Baig2015Community}.

%The community network clouds need to address the conditions of community networks and satisfy their requirements.
%We look at the different aspects of designing community cloud systems and the challenges involved below. %, and discuss the related work.
%
%% *** Section: Requirements
%%
%
%%% Section: REQUIREMENTS
%\section{Requirements and Challenges}
%\label{sec:requirements}
%
%In the following, requirements on the community clouds emphasised by different user communities are stated. 
%These requirements provide the foundation for the design of the community cloud system, and need to be satisfied for it to be deployed and adopted successfully by the community.
%
%\subsection{Security}
%There are many security challenges that need to be addressed for ensuring users' trust in the system. 
%With multiple independent cloud providers from the community, security becomes even more important in a community cloud. 
%The data and applications running on different cloud systems has to be protected from unauthorised access, damage and tampering.
%With the rise in sophisticated security threats and increase in data breaches, this is one of the most critical factor impacting any community cloud system.
% 
%
%\subsection{Autonomy}
%Community cloud systems may be formed based on individual cloud systems that are set up and managed independently by different owners. 
%%We cannot assume or require prior coordination or even trust between different cloud owners.
%%This means that each cloud owner can take decisions about his or her cloud setup without negotiating with other partners beforehand. 
%The main requirement for a cloud owner for participating in such a community cloud is that the local cloud setup should adhere to the common API provided by the community cloud. 
%In addition, the cloud owner should contribute some set of mutually agreed resources to the community.
%
%
%\subsection{Self-Management}
%Depending on the type of community, self-management capabilities may be an important requirement on the community cloud system. 
%%Community cloud nodes may join and leave the community cloud at any time.
%Community cloud should manage itself and continue providing services without disruption, even when part of the infrastructure run into issues. 
%Self-management should also help in the coordination between different cloud owners that become part of a federated community cloud.
%
%
%\subsection{Utility}
%For the acceptance of the community cloud, it should provide applications that are valuable for the community. 
%Usage strengthens the value of the community cloud, motivating its maintenance and update.
%
%
%%\subsection{Ease of Use}
%%Most of the users of the community cloud will not be proficient in cloud technologies, so setting up nodes for deployment and managing cloud software should be simple and straightforward.
%%Similarly, managing and updating the cloud software on the nodes should be as automatic as possible.
%
%
%\subsection{Incentives for Contribution}
%Some flavours of community clouds may be build upon collective efforts. 
%Such a community cloud builds on the contribution of the volunteers in terms of computing, storage, network resources, time and knowledge.
%For these community clouds to be sustainable, incentive mechanisms are needed to encourage users to actively contribute towards the system.
%
%
%\subsection{Support for Heterogeneity}
%The hardware and software used by members in a community cloud can have quite varying characteristics, which can result from the usage of different hardware, operating systems, cloud platforms or application software. 
%The community cloud system should handle this heterogeneity seamlessly.
%%There could be powerful machines with good network connectivity and abundant storage space.
%%On the other end, there are less powerful machines with limited CPU, RAM, disk space and bandwidth.
% 
%
%
%\subsection{Standard API}
%The cloud system should make it straightforward for the application programmers to design their applications in a transparent manner for the underlying heterogeneous cloud infrastructure.
%The API should provide the appearance of a middleware that obviates the need to customise the applications specific to each cloud architecture.
%This is essential for community clouds when these result from the federation of many independently managed clouds.
%Each such cloud may be using a different cloud management platform that may provide a different set of API.
%Providing a standard API for the community cloud ensures that applications written for one community cloud can also be deployed for another community cloud in the future and as they are integrated into the community cloud they can be easily deployed on new cloud architectures. 
%
%
%\subsection{QoS and SLA Guarantees}
%The community cloud system needs mechanisms for ensuring quality of service (QoS) and enforcing service level agreements (SLA).
%With business processes increasingly reliant on the cloud services and applications, businesses need strong guarantees before transitioning to a community cloud system.

%
%%% *** Research Subproblems
%%
%
%%\subsection{Research Subproblems of Community Cloud Computing}
%%\label{sec:subproblems}
%
%%We present the work that has not looked at the complete systems but tackled sub problems relevant for community clouds.
%
%%%\vspace{5pt}
%%We look at the different aspects of designing community cloud systems and the challenges involved, and discuss the related work.
%
%\subsubsection{Vision and Proposal}
%%\subsubsection{Community Cloud as Digital Ecosystems}
%Marinos et al.~\cite{Marinos2009} is one of the first work to offer a detailed vision for community cloud computing which they see as coming together of fields like green computing, volunteer computing, cloud computing and digital ecosystems.
%%
%%%\subsubsection{Credit Union Model}
%%Che et al.~\cite{Che2011} also present a vision of community cloud that was based on credit union model.
%%
%%%\subsubsection{Community Cloud}
%%Skadsem et al.~\cite{SkadsemKBM11} focus more on potential applications for communities by using local cloud services.
%%They have built distributed storage in P2P systems, and they want to extend that to the community cloud by incorporating virtualization. 
%%They envisage two usage scenarios for cloud applications in rural communities. 
%%First focuses on the distributed storage where they plan to provide an online message board for community for sharing photos and videos.
%%The other deals with compute-intensive operations such as collaborative editing of video footage.
%
%\subsubsection{Architecture and Design}
%Khan et al.~\cite{Khan2014Architecture} propose a collaborative distributed architecture for building a community cloud system that employs resources contributed by the members of the community network for provisioning infrastructure and software services. 
%Such architecture needs to be tailored to the specific social, economic and technical characteristics of the community networks for community clouds to be successful and sustainable. 
%Gall et al.~\cite{Gall2013} have explored how an InterCloud architecture~\cite{Buyya2010InterCloud} can be adapted to community clouds.
%%Esposito et al.~\cite{Esposito2013} present a flexible federated Cloud architecture 
%%based on a scalable publish and subscribe middleware for dynamic and transparent interconnection between different providers.
%
%\subsubsection{Economic Models for Resource Allocation}
%Community clouds need fair and efficient economic mechanisms for resource allocation and regulation for sustaining users' participation,
%for instance Zhao et al.~\cite{Zhao2014} explore efficient and fair resource sharing among the participants in community-based cloud systems.
%Other recent work has also addressed different issues in efficient and optimal resource allocation~\cite{Caton2014}. %~\cite{Chard2012, Haas2013, Punceva2013, Petri2013Broker}
%
%\subsubsection{Incentive Mechanisms}
%%Participants in a community cloud act independently and are not obliged to contribute. 
%To ensure sustainability and growth of the community cloud, incentive mechanisms are needed that encourage members to contribute with their hardware, effort and time.
%%When designing such mechanisms, the heterogeneity of the nodes and communication links has to be considered since each member brings in a widely varying set of resources and physical capacity to the system.
%Khan et al.~\cite{Khan2015Incentive} suggest an effort-based incentive mechanism for community cloud where effort is defined as contribution relative to the capacity of the users. %\cite{Khan2013TowardsIncentives, Khan2014Prototyping, Khan2015Incentive, Khan2014Macroeconomic}. 
%Social cloud~\cite{Chard2012, Caton2014} %\cite{Haas2011, Haas2013, Punceva2013} 
%employs virtual currency based mechanisms to encourage users to participate for gaining credits that they can use later to request resources from the system.
%%.... Incentives are needed to encourage user participation \cite{Khan2015Incentive,Khan2014Sparks,Khan2014Macroeconomic,Khan2014Prototyping,Khan2013TowardsIncentives,Buyuksahin2013} and \cite{Punceva2013}.
%
%\subsubsection{Security, Privacy and Trust}
%Security and privacy gain added significance in community clouds since users are contributing with their own resources to the system, and moreover, their data and applications are placed in other contributors' machines~\cite{Petri2012Trust}.
%%There has been research recently looking into the trust and security issues within the context of community clouds~\cite{Petri2012Trust, Baiardi2010}.


