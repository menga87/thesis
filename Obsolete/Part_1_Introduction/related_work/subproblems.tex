%
%
%%\subsection{Research Subproblems of Community Cloud Computing}
%%\label{sec:subproblems}
%
%%We present the work that has not looked at the complete systems but tackled sub problems relevant for community clouds.
%
%%%\vspace{5pt}
%%We look at the different aspects of designing community cloud systems and the challenges involved, and discuss the related work.
%
%\subsubsection{Vision and Proposal}
%%\subsubsection{Community Cloud as Digital Ecosystems}
%Marinos et al.~\cite{Marinos2009} is one of the first work to offer a detailed vision for community cloud computing which they see as coming together of fields like green computing, volunteer computing, cloud computing and digital ecosystems.
%%
%%%\subsubsection{Credit Union Model}
%%Che et al.~\cite{Che2011} also present a vision of community cloud that was based on credit union model.
%%
%%%\subsubsection{Community Cloud}
%%Skadsem et al.~\cite{SkadsemKBM11} focus more on potential applications for communities by using local cloud services.
%%They have built distributed storage in P2P systems, and they want to extend that to the community cloud by incorporating virtualization. 
%%They envisage two usage scenarios for cloud applications in rural communities. 
%%First focuses on the distributed storage where they plan to provide an online message board for community for sharing photos and videos.
%%The other deals with compute-intensive operations such as collaborative editing of video footage.
%
%\subsubsection{Architecture and Design}
%Khan et al.~\cite{Khan2014Architecture} propose a collaborative distributed architecture for building a community cloud system that employs resources contributed by the members of the community network for provisioning infrastructure and software services. 
%Such architecture needs to be tailored to the specific social, economic and technical characteristics of the community networks for community clouds to be successful and sustainable. 
%Gall et al.~\cite{Gall2013} have explored how an InterCloud architecture~\cite{Buyya2010InterCloud} can be adapted to community clouds.
%%Esposito et al.~\cite{Esposito2013} present a flexible federated Cloud architecture 
%%based on a scalable publish and subscribe middleware for dynamic and transparent interconnection between different providers.
%
%\subsubsection{Economic Models for Resource Allocation}
%Community clouds need fair and efficient economic mechanisms for resource allocation and regulation for sustaining users' participation,
%for instance Zhao et al.~\cite{Zhao2014} explore efficient and fair resource sharing among the participants in community-based cloud systems.
%Other recent work has also addressed different issues in efficient and optimal resource allocation~\cite{Caton2014}. %~\cite{Chard2012, Haas2013, Punceva2013, Petri2013Broker}
%
%\subsubsection{Incentive Mechanisms}
%%Participants in a community cloud act independently and are not obliged to contribute. 
%To ensure sustainability and growth of the community cloud, incentive mechanisms are needed that encourage members to contribute with their hardware, effort and time.
%%When designing such mechanisms, the heterogeneity of the nodes and communication links has to be considered since each member brings in a widely varying set of resources and physical capacity to the system.
%Khan et al.~\cite{Khan2015Incentive} suggest an effort-based incentive mechanism for community cloud where effort is defined as contribution relative to the capacity of the users. %\cite{Khan2013TowardsIncentives, Khan2014Prototyping, Khan2015Incentive, Khan2014Macroeconomic}. 
%Social cloud~\cite{Chard2012, Caton2014} %\cite{Haas2011, Haas2013, Punceva2013} 
%employs virtual currency based mechanisms to encourage users to participate for gaining credits that they can use later to request resources from the system.
%%.... Incentives are needed to encourage user participation \cite{Khan2015Incentive,Khan2014Sparks,Khan2014Macroeconomic,Khan2014Prototyping,Khan2013TowardsIncentives,Buyuksahin2013} and \cite{Punceva2013}.
%
%\subsubsection{Security, Privacy and Trust}
%Security and privacy gain added significance in community clouds since users are contributing with their own resources to the system, and moreover, their data and applications are placed in other contributors' machines~\cite{Petri2012Trust}.
%%There has been research recently looking into the trust and security issues within the context of community clouds~\cite{Petri2012Trust, Baiardi2010}.
