
% SECTION: Macroeconomic Policies
\subsection{Design of Social and Economic Policies}
\label{sec:design}

We discuss in this section the social and economic policies we propose for community clouds, addressing relevant issues of the technical, social, economic and legal aspects of the community cloud system~\cite{Khan2014Sparks,Khan2014Macroeconomic}. 
%We approach the problem by having explored some of the mechanisms previously in simulations~\cite{Khan2013TowardsIncentives} and also by developing a prototype implementation which is currently deployed in the Guifi community network~\cite{Jimenez2014} and which will allow to get users involved and participating in a real world scenario.

\begin{itemize}

\item Commons License
\\
The agreement and license to join a community cloud should encourage and help enforce reciprocal sharing for community clouds to work.
The Wireless Commons License~\cite{WCL2010} or Pico Peering Agreement~\cite{PicoPeering2005} is adopted by many community networks to regulate network sharing. 
This agreement could serve as a good base for drafting an extension that lays out the rules for community clouds.

\item Peering Agreements
When different community clouds federate together, agreements should ensure fairness for all the parties. 
Agreements between different communities should describe the rules for peering between clouds. 
Within such agreements, local currency exchanges could be extended to address cases of imbalance in contribution across different zones~\cite{Punceva2013}.


\item Ease of Use
\\
The easier it is for users to join, participate and manage their resources in the community cloud, the more the community cloud model will be adopted.
This requires lowering the startup costs and entry barriers for participation.
To this end, in terms of an institutional policy, we have developed a Linux-based distribution\footnote{\url{http://repo.clommunity-project.eu}}, to be used in the Guifi.net community cloud~\cite{Jimenez2014}. 
It will make the process of joining and consuming cloud services almost automated with little user intervention. 
This effect will make the community cloud appealing to non-technical users.

\item Social Capital
\\
Community clouds need to appeal to the social instincts of the community instead of solely providing economic rewards.
This requires maximising both bonding social capital~\cite{Coleman1988} within local community clouds in order to increase the amount of resources and commitment of the users, and bridging social capital in order to ensure strong cooperation between partners in federated community clouds. %For more http://www.bettertogether.org/ or http://bowlingalone.com/
Research on social cloud computing~\cite{Chard2012} has already shown how to take advantage of the trust relationships between members of social networks to motivate contribution towards a cloud storage service.


\item Transaction Costs
\\
The community cloud, especially in its initial stages, will require strong coordination and collaboration between early adopters as well as developers of cloud applications and services, so we need to lower the transaction costs for group coordination~\cite{Coase1937}.
This can take advantage of existing Guifi.net's mailing list\footnote{\url{http://guifi.net/en/forum}}, but also of the regular social meetings and other social and software collaboration tools.
It also requires finding the right balance between a strong central authority and decentralised and autonomous mode of participating for community members and software developers.


\item Locality
\\
Since the performance and quality of cloud application in community networks can depend a lot on the locality, applications need to be network and location aware, but this also requires that providers of resources should honour their commitment to local community cloud implying that most requests are fulfilled within the local zone instead of being forwarded to other zones.
We have explored the implications of this earlier when studying the relationship between federating community clouds~\cite{Khan2013TowardsIncentives, Khan2014Prototyping}.


\item Overlay Topology
\\
Community networks are an example of scale-free small-world networks~\cite{Vega2012}, and the community cloud that results from joining community networks users is expected to follow the same topology and inherit characteristics similar to scale-free networks. 
As the overlay between nodes in the community cloud gets created dynamically~\cite{Nakao2010}, the community cloud may evolve along different directions as users of the underlying community network join the system. 
As the applications in community cloud will most likely be location and network aware to make the most efficient use of the limited and variable resources in the network, the overlay steered concentration and distribution of consumers and providers of services direct the state and health of the community cloud.


\item Entry Barriers
\\
In order to control the growth of the community cloud and provide a reasonable quality of experience for early adopters and permanent users, different approaches can be considered, for example, a community cloud open to everyone, by invitation only, or one that requires a minimum prior contribution.


\item Role of Developers
\\
The developers of the cloud applications are expected to play an important intermediary role between providers of resources and consumers of services, for example adding value to the raw resources and selling them to consumers at a premium. 
End users could have both the roles of raw resource providers and consumers which find the value of the cloud in the provided applications.


\item Service Models
\\
Cloud computing offers different service levels, infrastructure, platform and software-as-a-service (SaaS). % (Iaas, Paas, and SaaS). 
Similar to the three economic sectors for provisioning goods, the third level, the SaaS of the cloud reaches the end users. 
For providing value from the beginning in the community cloud, we propose to prioritise provisioning SaaS at the early stage of the community cloud. %Read more http://en.wikipedia.org/wiki/Economic_sector


\item Value Addition and Differentiation
\\
The community cloud requires services that provide value for users. 
In addition, these services need to compete and differentiate from the generic cloud services available over the Internet.
In this line, FreedomBox\footnote{\url{http://freedomboxfoundation.org}} services focus on ensuring privacy, and FI-WARE CoudEdge\footnote{\url{http://catalogue.fi-ware.eu/enablers/cloud-edge}} and ownCloud\footnote{\url{http://owncloud.org}} let cloud applications consume resources locally.

\end{itemize}

%%%%%%%%%%%%%%%%%%%%%%%%%%%%%%%%%%%%%%%%%

%% >>> A more summarised account of mechanisms (taken from INCoS paper)
% SECTION: Macroeconomic Policies
%\subsection{Socio-Economic Mechanisms for Supporting Collaboration}
%\label{sec:economic-mechanisms}
%
%The purpose of economic mechanisms and social and psychological incentives is to let the community cloud transition from inception through early adoption to finally ubiquitous usage~\cite{Khan2014Macroeconomic}.
%In the nascent stage, the community cloud may not be able to provide a lot of value until a critical mass of users are using the system.
%After that threshold, still the relative cost to achieve a little utility will be significant, which means that the early adopters of the system remain highly motivated and committed to the success of community cloud and continue to contribute resources even though they receive little value from the system in return.
%But once a significant proportion of community network members have joined the community cloud, the relative cost to obtain value from the system tumbles and in the longer run the system is able to sustain itself with contributions that may be small in size but are made by a large number of users.
%
%The mechanisms must take into account the costs and benefits involved in participating in community cloud.
%For instance, the initial costs for setting up nodes in the community cloud involves hardware and installation costs.
%The continuous operation of the cloud node requires additional costs including network costs given by donating network bandwidth and any other subscription fees, energy costs to pay for electricity bills to run the computer equipment as well as cooling apparatus, maintenance cost to fund any technical support and replacements of parts, and hosting costs to provide storage space for the equipment.
%Besides these costs at the individual level, there are also the transaction costs and management overheads necessary for the collective operation of community cloud.
%
%The individuals in community cloud act as private enterprises where they offer services to generate revenue. 
%The revenue for the community cloud users include tangible benefits like the services and applications that they will be able to consume, and intangible benefits like the sense of belonging to the community and personal satisfaction because of their contributions.
%The services can range from infrastructure to platform to software services meeting a spectrum of different needs of the users.
%
%Different policies addressing relevant issues of the technical, social, economic and legal aspects of the community cloud are designed to encourage collaboration, for example commons license and peering agreements can be implemented that extend the idea of reciprocal sharing from Wireless Commons License\footnote{\url{http://guifi.net/es/ProcomunXOLN}} and Pico Peering Agreement\footnote{\url{http://www.picopeer.net}} in community networks.
%The social context of community networks provides opportunity to harness social capital and the different roles of social relationships.
%Similarly, 
%	lowering transaction costs and entry barriers, 
%	facilitating participation of developers, 
%	exploring different service models to provide value addition and differentiation,
%	and taking advantage of locality and overlay topology of the network
%can prove useful.
%Such mechanisms help adapt the ecosystem of community cloud infrastructure and services to the aspirations of community network members.
%
%
