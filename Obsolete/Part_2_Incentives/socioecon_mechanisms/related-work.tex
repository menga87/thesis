%
%
%% Section: RELATED WORK
%\section{Related Work}
%\label{sec:related-work}
%
%The idea of collaboratively built community clouds follows on from earlier distributed voluntary computing platforms, like 
%	BOINC~\cite{Anderson2004},
%	Folding@home~\cite{Beberg2009}, 
%	HTCondor~\cite{Thain2005}, 
%	PlanetLab~\cite{Chun2003}
%	and Seattle~\cite{Cappos2009}, 
%which mainly rely on altruistic contribution of resources from the users, 
%though various mechanisms have been studied in the context of peer-to-peer systems~\cite{Shen2010} 
%that address different problems of collaborative resource sharing. 
%There are only a few research proposals for community cloud computing~\cite{Marinos2009}. 
%Most of them do not go beyond the level of an architecture, and at most a practical implementation is presented.
%None of these implementations, to our knowledge, are actually being deployed inside of real community networks. 
%
%The Cloud@Home\cite{Distefano2012} project aims to harvest in resources from the community for meeting the peaks in demand, working with public, private and hybrid clouds to form cloud federations.
%The authors propose a rewards and credit system for ensuring quality of service.
%Gall et al.~\cite{Gall2013} have explored how an InterCloud architecture~\cite{Buyya2010InterCloud} can be adapted to community clouds.
%Social cloud computing~\cite{Chard2012} takes advantage of the trust relationships between members of social networks to motivate contribution towards a cloud storage service. 
%Users trade their excess capacity to earn virtual currency and credits that they can utilize later, and consumers submit feedback about the providers after each transaction which is used to maintain reputation of each user.
%Social clouds have also been deployed in CometCloud framework by federating resources from multiple cloud providers~\cite{Punceva2013}.
%Zhao et al.~\cite{Zhao2014} explore efficient and fair resource sharing among the participants in community-based cloud systems.
%Jang et al.\cite{Jang2014} implement personal clouds that combine local, nearby and remote cloud resources to enhance the services available on mobile devices.
%
%From the review of related work, we find that none of the above cases correspond to the concrete situation of community networks such as targeted by us. 
%In the cloud system that we propose, we aim to take into account several of the important factors that characterize community networks, such as the scenarios we identified from the conditions of community networks.
