%
%
%% Section: CONCLUSION
%\section{Conclusion and Future Work}
%\label{sec:conclusion}
%
%Community clouds take advantage of resources available within community networks for realising cloud-based services and applications tailored to local communities.
%Being community clouds a case of private provisioning of public goods, economic mechanisms and policies are needed to direct their growth and sustainability. 
%First, we analysed the key socio-technical characteristics of community networks in and presented two community cloud scenarios, the local community cloud and the federated community cloud. 
%Secondly, we identified the cost and value evolution of the community cloud during its emergence and under permanent operation. 
%A core number of highly motivated contributors is needed at the beginning. 
%Once the community cloud is operational, its value should easily exceed the cost of the minor contribution expected from the users. 
%The socio-economic context of community networks forms the basis for the social, technical and economic policies that we proposed for community clouds. 
%We outlined and illustrated these policies that address technical, social, economic and legal aspects of the community cloud system.
%
%Based on the proposed socio-economic policies, our next step is to design and integrate them in our prototype implementation of the community cloud that we currently deploy in the real-world Guifi.net community network. 
%The resulting empirical studies will help assessing the effect of the proposed economic mechanisms further.
%Our hope is that community clouds will complement the existing public cloud services paving the way for innovative and interesting applications for local communities. 
%
