%
%
%% Section: RELATED WORK
%\section{Related Work}
%\label{sec:related-work}
%
%The idea of collaboratively built community clouds follows on from earlier distributed voluntary computing platforms, like 
%	BOINC~\cite{Anderson2004},
%	Folding@home~\cite{Beberg2009}, 
%%	HTCondor~\cite{Thain2005}, 
%	PlanetLab~\cite{Chun2003}
%	and Seattle~\cite{Cappos2009}, 
%which mainly rely on altruistic contribution of resources from the users, though various mechanisms have been studied in the context of peer-to-peer systems~\cite{Shen2010} that address different problems of collaborative resource sharing. 
%There are only a few research proposals for community cloud computing, for example Cloud@Home\cite{Distefano2012} project aims to harvest in resources from the community for meeting the peaks in demand, working with public, private and hybrid clouds to form cloud federations.
%Social cloud computing~\cite{Chard2012} takes advantage of the trust relationships between members of social networks to motivate contribution towards a cloud storage service, and such social clouds have also been deployed in CometCloud framework by federating resources from multiple cloud providers~\cite{Punceva2013}.
%Gall et al.~\cite{Gall2013} have explored how an InterCloud architecture~\cite{Buyya2010InterCloud} can be adapted to community clouds, and federated cloud architectures~\cite{Moreno-Vozmediano2012} in general are being actively explored for combining services from multiple cloud providers.
%
%From the review of related work, we find that none of the above cases have a prototype for the concrete situation of the community networks. 
%In the cloud system that we present, we aim to take into account several of the important social and technical factors that characterise community networks, and therefore the cloud architecture we propose is tailored to the specific context of the community networks.