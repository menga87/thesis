

% Section: CONCLUSION
\section{Summary}
\label{sec:conclusion}

Community clouds take advantage of resources available within community networks for realising cloud-based services and applications tailored to local communities.
Being community clouds a case of private provisioning of public goods, economic mechanisms and policies are needed to direct their growth and sustainability. 
First, we analysed the key socio-technical characteristics of community networks in and presented two community cloud scenarios, the local community cloud and the federated community cloud. 
Secondly, we identified the cost and value evolution of the community cloud during its emergence and under permanent operation. 
A core number of highly motivated contributors is needed at the beginning. 
Once the community cloud is operational, its value should easily exceed the cost of the minor contribution expected from the users. 
The socio-economic context of community networks forms the basis for the social, technical and economic policies that we proposed for community clouds. 
We outlined and illustrated these policies that address technical, social, economic and legal aspects of the community cloud system.
%
%
%
%Community networks would greatly benefit from the additional value provided by the applications and services deployed in the community clouds. 
%Such clouds for community networks, however, have not been specified yet by the related work to enable further developments.
We proposed a collaborative distributed service architecture for providing cloud services that is tailored to the unique nature and conditions of community networks. 
Our architecture proposes on top of existing cloud management platforms a set of support services for regulated resource sharing to encourage active participation of the community members which is required to form and maintain the cloud infrastructure, and for supporting the federation of cloud resources. 
%Since community networks are volunteer organisations, we consider such support services an essential step for assuring a sustainable community cloud within community networks. 
%We have deployed attractive applications on community cloud infrastructures in the Guifi.net community network to assess the applications' performance. We observed the feasibility of such applications in the community cloud and their correct functioning, which is crucial to attract real users for the next step of our research. 

%Based on the proposed architecture, our next step is to further develop the identified components and test them in the community cloud by engaging end users from community networks with these applications. 
%The deployed prototype will allow running experiments in the real setting of a community network to investigate the performance of such a collaborative distributed community cloud. 
%The resulting empirical studies will feed back to validate and improve the design of different components in the architecture. 
%This proposal of the distributed community network architecture is a first step to exploit the potential of community clouds to complement existing public cloud services, opening the way to build collaborative user-shaped innovative applications for local communities. 
