
% Section: INTRODUCTION
\section{Introduction}
\label{sec:introduction}

Community networks aim to satisfy a community's demand for Internet access and services using open unlicensed wireless spectrum and off-the-shelf communication equipment. 
Most community networks originated in rural areas which commercial telecommunication operators left behind when focusing the deployment of their infrastructure on urban areas. 
The lack of broadband access brought together different stakeholders of such geographic areas to team up and invest, create and run a community network as an open telecommunication infrastructure based on self-service and self-management by the users~\cite{Elianos2009}.

These community networks are a real world example of a collective that shares information and communication technology (ICT) infrastructure and human resources. 
The ICT resources shared are the bandwidth of the wireless network formed by the networking hardware belonging to multiple owners. 
This bandwidth allows members of the community network obtaining access to the Internet or use services and applications inside of the community network.
The human resources shared are the time and knowledge of the participants, needed to maintain the network and technically organise it for further growth. 

Sharing of network bandwidth has early been identified as essential and is part of the membership rules or peering agreements of many community networks, which regulate the usage and growth of the network. 
The Wireless Commons License (WCL)~\cite{WCL2012} of many community networks states that the network participants that extend the network, e.g. contribute new nodes, will extend the network in the same WCL terms and conditions, allowing traffic of other members to transit on their own network segments. Since this sharing is done by all members, community networks successfully operate as IP networks. 

Today's Internet, however, is more than bandwidth resources. 
Computing and storage resources are shared through Cloud Computing, offering virtual machine instances over infrastructure services, APIs and support services through platform-as-a-service, and Web-based applications to end users through software-as-a-service. 
These services, now common practice in today's Internet, hardly exist in community networks~\cite{Khan2013Clouds}. 
Services offered in community networks still run on machines exclusively dedicated to a single member. 
Community network members, however, do use commercial cloud solutions, for instance for network administration, where sometimes a commercial storage service is used for node data. 
Why have clouds not emerged inside of the community networks?

We argue that community cloud, a cloud infrastructure formed by community-owned computing and communication resources, has many technical and social challenges so that the main drivers of today's contribution to community networks, voluntariness and altruistic behaviour, are not enough to successfully cope with it. 
Our hypothesis is that for community cloud to happen, the  members' technical and human contribution needed for such a cloud, needs to be steered by incentive mechanisms that pay back the users' contribution with a better quality of experience for them.

In this paper, we present an incentive mechanism tailored to community networks.
The main contributions of this paper are the following:
\begin{enumerate}
    \item From the analysis of the key socio-technical characteristics of community networks, we identify two scenarios for community clouds, the local clouds and federated clouds, for which a community cloud management system is proposed.
    \item We design an incentive mechanism that is part of the community cloud architecture and evaluate its behaviour in simulations of community cloud scenarios. 
\end{enumerate}


We elaborate our contributions in the following way: In section~\ref{sec:design} we present our system model and design.
In section~\ref{sec:evaluation}, we evaluate our incentive mechanism in a community cloud scenario.
In section~\ref{sec:related-work} we relate the work of other authors with our results. We discuss open issues in section 5 on future work and in section~\ref{sec:conclusion} we conclude our findings.
