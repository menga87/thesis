

% Section: DISCUSSION
\section{Future Work}
\label{sec:future_work}

We have investigated incentive mechanisms for community clouds based on reciprocal resource sharing. 
Our results indicate their impact on the efficiency of the system and on regulating the resource assignments. 
The understanding gained from the different experimental results helps in the design of the policies that such incentive mechanism could follow in a future prototype of real community cloud system. 

Our results, however, have revealed new issues that are to be addressed in the next steps towards a real cloud system. 
First, we have not yet investigated the behaviour of the incentive mechanism for extended periods of time. 
Further experiments are needed to study how the mechanism can be used for long durations.
Secondly, we have not yet investigated the incentive mechanism in a prototype deployed in a real community network.

For the permanent operation of the cloud system with the incentive mechanism, the mechanism needs to be able to adapt to the system state in runtime. 
The mechanism will need to be able to take into account the evolution of the system with regards to users, resources, and different kind of behaviours. 
Therefore, parameters of the incentive mechanism will need to be defined as functions of the system state in order to account and decide correctly on the current situation. 
In order to further develop this runtime adaptability, a two-fold approach, which on one hand extends the simulations with refined system models and on the other hand evaluates the performance of deployed prototype components, is suggested to assure the realisation of an operative adaptive system.  

A prototype of the incentive mechanism integrated in a cloud management platform is needed to be able to obtain performance results from real users and services. 
An operative modular system is needed that allows an easy modification of its components according to the simulation results. 
The transfer of the simulation results to the deployed system should be required, in order to assure that the simulated system model reflects the real system, and that the obtained findings can actually be brought into the real system in a feasible way.      

Finally, the deployment of several federated clouds with real users and real usage should ultimately be undertaken. 
Such large-scale cloud deployments need to have an extended implementation of a communication middleware for the coordination in a network of super nodes, complemented by additional services, to fully achieve an incentive-based resource assignment. 
For such systems, additional work is needed to develop in detail the feedback loop between the user's contribution and the experience the user obtains from the cloud services, needed for the building and maintenance of a cloud in community networks. 
