% Based on LLNCS.DEM the demonstration file of the LaTeX macro package from Springer-Verlag for LNCS version 2.4 for LaTeX2e as of 16. April 2010
%
% Moved LNCS Template's code to a separate file
\input{LNCSTemplate/header.tex}
%
%\usepackage{makeidx}  % allows for indexgeneration
%
\begin{document}
%
\frontmatter          % for the preliminaries
%
\pagestyle{headings}  % switches on printing of running heads
\addtocmark{Hamiltonian Mechanics} % additional mark in the TOC
%
%
%
\mainmatter              % start of the contributions
%
\title{Towards Incentive-based Resource Assignment and Regulation in Clouds for Community Networks}
%
\titlerunning{Resource Regulation in Clouds for Community Networks}  
% abbreviated title (for running head) also used for the TOC unless \toctitle is used
%

\author{ Amin M. Khan \and \"Umit C. B\"uy\"uk\c{s}ahin \and Felix Freitag }
%
\authorrunning{A. M. Khan, \"U. C. B\"uy\"uk\c{s}ahin, and F. Freitag} % abbreviated author list (for running head)
%
%%%% list of authors for the TOC (use if author list has to be modified)
\tocauthor{Amin M. Khan, \"Umit C. B\"uy\"uk\c{s}ahin, and  Felix Freitag}
%
\institute{
    Department of Computer Architecture\\
    Universitat Polit\`ecnica de Catalunya, Barcelona, Spain\\
    \email{ \{mkhan, ubuyuksa, felix\}@ac.upc.edu }\\
}

\maketitle              % typeset the title of the contribution

\begin{abstract}
Community networks are built with off-the-shelf communication equipment aiming to satisfy a community's demand for Internet access and services. These networks are a real world example of a collective that shares ICT resources. But while these community networks successfully achieve the IP connectivity over the shared network infrastructure, the deployment of applications inside of community networks is surprisingly low. Given that community networks are driven by volunteers, we believe that bringing in incentive-based mechanisms for service and application deployments in community networks will help in unlocking its true potential. We investigate in this paper such mechanisms to steer user contributions, in order to provide cloud services from within community networks. From the analysis of the community network's topology, we derive two scenarios of community clouds, the local cloud and the federated cloud. We develop an architecture tailored to community networks which integrates the incentive mechanism we propose. In simulations of large scale community cloud scenarios we study the behaviour of the incentive mechanism in different configurations, where slices of homogeneous virtual machine instances are shared. Our simulation results allow us to understand better how to configure such an incentive mechanism in a future prototype of a real community cloud system, which ultimately should lead to realisation of clouds in community networks.

\keywords{incentive mechanisms, cloud computing, community networks, distributed resource sharing}
\end{abstract}
%

%% *** Introduction

% Section: INTRODUCTION
\section{Introduction}
\label{sec:introduction}

Community networks aim to satisfy a community's demand for Internet access and services using open unlicensed wireless spectrum and off-the-shelf communication equipment. 
Most community networks originated in rural areas which commercial telecommunication operators left behind when focusing the deployment of their infrastructure on urban areas. 
The lack of broadband access brought together different stakeholders of such geographic areas to team up and invest, create and run a community network as an open telecommunication infrastructure based on self-service and self-management by the users~\cite{Elianos2009}.

These community networks are a real world example of a collective that shares information and communication technology (ICT) infrastructure and human resources. 
The ICT resources shared are the bandwidth of the wireless network formed by the networking hardware belonging to multiple owners. 
This bandwidth allows members of the community network obtaining access to the Internet or use services and applications inside of the community network.
The human resources shared are the time and knowledge of the participants, needed to maintain the network and technically organise it for further growth. 

Sharing of network bandwidth has early been identified as essential and is part of the membership rules or peering agreements of many community networks, which regulate the usage and growth of the network. 
The Wireless Commons License (WCL)~\cite{WCL2012} of many community networks states that the network participants that extend the network, e.g. contribute new nodes, will extend the network in the same WCL terms and conditions, allowing traffic of other members to transit on their own network segments. Since this sharing is done by all members, community networks successfully operate as IP networks. 

Today's Internet, however, is more than bandwidth resources. 
Computing and storage resources are shared through Cloud Computing, offering virtual machine instances over infrastructure services, APIs and support services through platform-as-a-service, and Web-based applications to end users through software-as-a-service. 
These services, now common practice in today's Internet, hardly exist in community networks~\cite{Khan2013Clouds}. 
Services offered in community networks still run on machines exclusively dedicated to a single member. 
Community network members, however, do use commercial cloud solutions, for instance for network administration, where sometimes a commercial storage service is used for node data. 
Why have clouds not emerged inside of the community networks?

We argue that community cloud, a cloud infrastructure formed by community-owned computing and communication resources, has many technical and social challenges so that the main drivers of today's contribution to community networks, voluntariness and altruistic behaviour, are not enough to successfully cope with it. 
Our hypothesis is that for community cloud to happen, the  members' technical and human contribution needed for such a cloud, needs to be steered by incentive mechanisms that pay back the users' contribution with a better quality of experience for them.

In this paper, we present an incentive mechanism tailored to community networks.
The main contributions of this paper are the following:
\begin{enumerate}
    \item From the analysis of the key socio-technical characteristics of community networks, we identify two scenarios for community clouds, the local clouds and federated clouds, for which a community cloud management system is proposed.
    \item We design an incentive mechanism that is part of the community cloud architecture and evaluate its behaviour in simulations of community cloud scenarios. 
\end{enumerate}


We elaborate our contributions in the following way: In section~\ref{sec:design} we present our system model and design.
In section~\ref{sec:evaluation}, we evaluate our incentive mechanism in a community cloud scenario.
In section~\ref{sec:related-work} we relate the work of other authors with our results. We discuss open issues in section 5 on future work and in section~\ref{sec:conclusion} we conclude our findings.


%% *** Section: System Model, Architecture and Design

% Section: DESIGN
\section{The Distributed Auctioneer}
\label{sec__dist_auctioneer_design}

We propose a framework for devising distributed 
protocols executed by the providers that correctly simulate the auctioneer.
The framework is sufficiently general to simulate different auctions.
To illustrate its applicability, we provide two implementations 
of the framework for standard and double 
bandwidth allocation auctions, respectively.
We describe the framework in two steps. First, we provide a general definition
where we do not specify the details about how to implement the simulation
of the algorithm $\A$. Then, we describe how to simulate $\A$
by leveraging parallelism to speed up its execution.


\subsection{General Framework}
The input of the framework at each provider $j$ is a vector $\vec{b}^j$
of bids submitted to $j$ and the output is either a pair $(x,\vec{p})$ containing
an allocation $x$ and a vector of payments $\vec{p}$ or the special value $\bot$.
As illustrated in Figure~\ref{fig:framework}, the framework chains the execution
of two building blocks: \emph{bid agreement} and \emph{allocator}.
Each provider $j$ inputs $\vec{b}^j$ to the bid agreement,
which outputs either a vector $\vec{b}$ or $\bot$.
In the former case, $j$ inputs $\vec{b}$ to the allocator.
If all providers follow the protocol, then the bid agreement ensures
that they all output some vector $\vec{b}$ containing all valid bids,
and the allocator ensures that they all output a pair $(x,\vec{p})$
with probability determined by $\A$.

\begin{figure}[tbp]
	\centering
	\includegraphics[width=0.75\columnwidth,keepaspectratio]{basicprotocol}
	\caption{Framework: Bid Agreement (BA) and Allocator (A)}
	\label{fig:framework}
\end{figure}

In the following paragraphs, we describe each block 
in more detail by defining properties that must be satisfied 
by any implementation of the block, and then show in the analysis that
every implementation of the framework is $k$-resilient and correctly simulates
the auctioneer based only on the properties of the blocks. This makes
the proof independent from the actual implementation.
In all blocks, an implementation $P$ must satisfy
the property of \emph{$k$-resiliency for solution preference}, i.e.,
$P$ must be a $k$-resilient equilibrium, under
the assumption that players have preference for a solution 
and number of agents not in the same coalition is sufficiently high.
Specifically, the output of every block is either some valid value or $\bot$.
We can split the set of outcomes of the block (combinations of outputs)
into the set $A$ of solutions where all providers output the same valid value
and the set $B$ of remaining outcomes. 
In a correct execution, we want the outcome to lie in $A$.
To ensure this and that the protocol is a $k$-resilient equilibrium,
we need to assume that providers obtain a higher utility for outcomes in $A$
than for outcomes in $B$ (preference for a solution), and $m > f(k)$ for
some function $f$ defined for every $k > 0$.
The assumption of preference for a solution of the framework
is equivalent to providers preferring to receive the payments.

\subsubsection{Bid Agreement}
The input at provider $j$ is the vector $\vec{b}^j$ of bids sent to $j$.
The output is a vector $\vec{b}$ or the special value $\bot$.
In addition to $k$-resiliency for solution preference,
this block must ensure two conditions when all providers follow the protocol: 
(1) \emph{eventual agreement}, defined as
all providers eventually outputting the same vector $\vec{b}$,
and (2) \emph{validity}, defined as,
for every bidder $i$ that submits the same bid $b_i'$
to all providers, the output at every provider is $b_i = b_i'$.

\begin{property}
\label{prop:bid-agreement}
A protocol $P$ implements bid agreement if and only if it satisfies two conditions:
(1) if all providers follow $P$, then $P$ satisfies eventual agreement and validity;
and (2) $k$-resiliency for solution preference.
\end{property}

If we can assume that the bids of malicious bidders are obtained from a finite
set of values and are equally likely, then 
a suitable approach is to use the rational consensus protocol proposed in~\cite{Afek2014},
which has inputs $\{0,1\}$ and outputs in $\{0,1,\bot\}$,
and satisfies the following two properties:
(a) if all providers follow the protocol, then all providers eventually
output the same bit, which is input by some provider;
and (b) $k$-resiliency for solution preference, assuming $m > 2k$
and that the input of every provider not in the same coalition
is either the same value or is $0$ or $1$ with equal probability.
This protocol can be used to implement the bid agreement as follows.
For each bidder $i$, provider $j$ generates a stream of bits uniquely
determined from $b_i^j$ and inputs each bit to a rational consensus instance;
if some instance outputs $\bot$, then $j$ outputs $\bot$,
otherwise, $j$ converts the stream to a bid $b_i$ and outputs a bid $b_i^*$,
where $b_i^* = b_i$ if $b_i$ is valid, or $b_i^*$ is some pre-determined valid bid otherwise.
To distinguish between different instances of rational consensus,
providers may append to the messages of each instance
the identifier of each bidder and the position of each bit.
Clearly, providers only output valid bids or the value $\bot$.
By (a), if all providers follow the protocol, then
eventual agreement and validity hold, showing (1).
Condition~(2) follows directly from (b) and $m>2k$ if
the input of every provider satisfies the assumptions of (b).
To see why these assumptions are true, notice
that, for each bidder $i$, if $i$ is not malicious, 
the input of all providers not in the same coalition is $i$'s true bid,
and if $i$ is malicious, then the bid $b_i^j$ sent by $i$ to $j$ is uniformly
distributed. If the set of possible bids is the set of all integers, 
then the stream of bits obtained from $b_i^j$ is also random.
These are reasonable assumptions, 
since we expect the behavior of malicious bidders to be arbitrary.

\subsubsection{Allocator} The input at every provider is a vector $\vec{b}$ of bids,
and the output is either a pair $(x,\vec{p})$ or $\bot$.
We want the allocator to satisfy four conditions. First, we want the allocator
to correctly simulate $\A$, i.e., given that
all providers input the same vector $\vec{b}$ and follow the protocol,
every provider must eventually output pair $(x,\vec{p})$
with probability $\A(x,\vec{p} \mid \vec{b})$.
Second, we want resilience to collusive influences,
defined as, for all coalitions $K$ of at most $k$ elements,
if all providers not in $K$ input $\vec{b}$
and follow the protocol, then no $j \notin K$ outputs a pair $(x,\vec{p})$
with probability higher than $\A(x,\vec{p} \mid \vec{b})$,
regardless of the protocol followed by providers in $K$.
Intuitively, no coalition $K$ can influence the output of providers not in $K$,
except that they may output $\bot$ with higher probability.
Third, we want input validation to ensure that
providers have preference for solutions at the bid agreement.
More precisely, if two providers input different vectors
and follow the protocol, then they both output $\bot$,
regardless of the protocol followed by other providers.
Finally, we want $k$-resilience for solution preference
given that all providers have the same input.

\begin{property}
\label{prop:allocator}
A protocol $P$ implements the allocator if and only if it satisfies four conditions:
(1) correct simulation of $\A$; (2) resilience to collusive influence;
(3) input validation; and (4) $k$-resiliency for solution preference
if all providers have the same input.
\end{property}

We discuss implementations of the allocator in \Cref{sec__distrib_auctioneer_protocol_implementation}.

\subsubsection{Analysis}
We show in Theorem~\ref{theorem:simulation}
that a protocol that implements our framework correctly
simulates the auctioneer and is $k$-resilient.

\begin{theorem}
\label{theorem:simulation}
For every protocol $P$ that implements the framework,
$P$ correctly simulates the auctioneer,
and there exists a function $f$ such that, if $m > f(k)$,
then $P$ is a $k$-resilient equilibrium.
\end{theorem}

%\subsection{Proof of Theorem~\ref{theorem:simulation}}
\begin{proof}
\label{app:simulation}
First, we show that $P$ correctly simulates the auctioneer.
Every provider $j$ inputs $\vec{b}^j$ to the bid agreement.
By (1) of Property~\ref{prop:bid-agreement}, regardless of the inputs,
all providers output the same vector $\vec{b}$
that satisfies validity. By (1) of Property~\ref{prop:allocator},
the outcome of the simulation is pair $(x,\vec{p})$
with probability $\A(x,\vec{p} \mid \vec{b})$.
This concludes the first step of the proof.

Now, we show that $P$ is a $k$-resilient equilibrium for $m > f(k)$ for some $f$.
Fix a coalition $K$. We take $f$ to be larger for all $k$ than the minimum value of $m$
required by Properties~\ref{prop:bid-agreement} and~\ref{prop:allocator}.
These properties imply that, if providers have preference for a solution
at the bid agreement, then the implementations of bid agreement is $k$-resilient,
so providers in $K$ prefer to follow $P$ for bid agreement.
Since this guarantees that all providers have the same input at the
allocator, the implementation of the allocator is also $k$-resilient,
implying that $P$ is $k$-resilient.

Now, we show that solution preference holds for both blocks.
Recall that the outcome is not $\bot$ only if
providers not in $K$ output the same pair,
and, if the outcome is $\bot$, then the utility is $0$.
Hence, providers in $K$ prefer (obtain an expected utility at least as high)
that providers not in $K$ output the same pair $(x,\vec{p})$;
in this case, they clearly prefer to output $(x,\vec{p})$ as well, thus they have preference
for a solution at the allocator. Now, consider the bid agreement.
The utility of an outcome of this block is the expected
utility given that providers not in $K$ follow $P$ and providers in $K$
follow an arbitrary protocol. Clearly, providers in $K$
prefer that no provider in $K$ outputs $\bot$.
By (3) of Property~\ref{prop:allocator}, providers in $K$
prefer that all providers not in $K$ output the same vector.
By (2) of Property~\ref{prop:allocator},
providers in $K$ cannot increase the probability of any outcome of the framework 
other than $\bot$ by deviating, thus, they cannot increase their expected utility by outputting
a vector $\vec{b}' \neq \vec{b}$ at the bid agreement.
This shows that providers have preference for a 
solution at the bid agreement, concluding the proof.
\end{proof}

%\subsection{Framework for Distributed Auctioneer}
\subsection{Parallel Allocator Framework}
\label{sec__distrib_auctioneer_protocol_implementation}

We describe a framework for implementations of
the allocator that satisfy Property~\ref{prop:allocator}.
We explore the possibility of parallelising the execution of $\A$ in multiple providers.
Although this approach introduces the overhead of communication between providers,
since $\A$ is often computationally intensive,
its parallelisation compensates for this overhead.

\begin{figure}[tbp]
	\centering
	\includegraphics[width=0.5\columnwidth,keepaspectratio]{taskdecomposition}
	\caption{Decomposition of the Allocator into Tasks}
	\label{fig:taskdecomposition}
\end{figure}

The framework consists in an initial invocation of a 
building block for \emph{input validation} followed
by the simulation of $\A$, which invokes two additional building blocks:
\emph{data transfer} and \emph{common coin}.
The input is a vector of bids and the output is either $\bot$ or a pair $(x,\vec{p})$.
At the invocation of each block, providers either output a valid value
or $\bot$; in the latter case, they output $\bot$ at the allocator.
To describe the simulation of $\A$,
it is useful to characterise the execution of $\A$
in terms of a graph of tasks, where nodes correspond to
tasks to be executed in sequence and edges 
represent data dependencies. 
This graph establishes a partial order
of tasks; every two tasks that are not ordered can be executed in parallel
by different providers. Figure~\ref{fig:taskdecomposition}
gives an example of a graph of $4$ tasks,
where tasks T2.1 and T2.2 can be executed in parallel.
To cope with collusion, each task $T$ is assigned to a set $S$ of at least $k+1$ providers.
If a task $T'$ is to be executed by a set $O \neq S$ of providers and $T'$ depends on
the result of $T$, then the providers of $S$ transfer data to the providers of $O$
using the data transfer building block. In a correct simulation of $\A$,
there must be one final task that depends on all other tasks,
where all providers gather all the required data to produce the final output.
Whenever providers need a random number distributed according to a 
probability distribution $\Pi$, they invoke the common coin with input $\Pi$.
Figure~\ref{fig:parallel-allocator} illustrates the framework
for the task decomposition of Figure~\ref{fig:taskdecomposition}.

\begin{figure}[tbp]
	\centering
	\includegraphics[width=0.70\columnwidth,keepaspectratio]{finalprotocol}
	\caption[Parallel Allocator]{Parallel Allocator: Input Validation (IV), Data Transfer (DT), and Common Coin (CC)}
	\label{fig:parallel-allocator}
\end{figure}


As in the previous section, we describe properties that must be satisfied
by the implementations of each block and then show that every implementation
of this framework satisfies Property~\ref{prop:allocator}.

\subsubsection{Input Validation}
The input is a vector $\vec{b}$ and the output is either
$\bot$ or $\vec{b}$. We want an implementation to satisfy 
$k$-resiliency for solution preference and that all providers 
eventually output $\vec{b}$ given that they all input $\vec{b}$,
and we need to satisfy (3) from Property~\ref{prop:allocator}.

\begin{property}
\label{prop:iv}
An implementation $P$ of the input validation must satisfy three conditions:
(1) if two providers follow $P$ and have different inputs, then they eventually output $\bot$;
(2) if all providers follow $P$ with the same input $\vec{b}$, then they eventually output $\vec{b}$;
and (3) $k$-resiliency for solution preference if all providers have the same input.
\end{property}

A simple implementation is to have providers broadcasting their vectors of bids
and outputting $\bot$ when two different vectors are detected.
This clearly satisfies (1) and (2),
whereas (3) is immediately true if providers have preference for a solution and $m > k$.

\subsubsection{Common Coin}
The input is a probability distribution $\Pi$
and the output is either $\bot$ or a number distributed according to $\Pi$.
Given that all providers have the same input, we 
want the common coin to satisfy $k$-resilience for solution preference 
and to output the same random number.

\begin{property}
\label{prop:common-coin}
Given that all providers have input $\Pi$,
an implementation $P$ of the common coin must satisfy two conditions:
(1) if all providers follow $P$, then they eventually output 
the same value distributed according to $\Pi$;
and (2) $k$-resiliency for solution preference.
\end{property}

A possible implementation of the shared coin is the protocol from~\cite{Abraham2013}.
The idea is that every provider $j$ commits to a random number $r_j \in [0,1]$,
before learning the random numbers of every other provider not in its coalition.
Then, providers reveal all random numbers and compute the output by summing all numbers modulo $1$.
If some provider $j$ sees a number not in $[0,1]$ or some provider does not send
a random value compatible with its commitment, then it outputs $\bot$.
Otherwise, $j$ applies a transformation on the computed value, which is uniformly distributed in $[0,1]$,
to produce an output that is distributed according to $\Pi$.

It is clear that all providers output the same random number distributed according to the common input $\Pi$
if they follow the protocol. Assuming that $m > k$, no provider $j$ can manipulate
the probability distribution of the output by not committing to $r_j$ selected at random
without some provider outputting $\bot$, even if $j$ is in a coalition of at most $k$ providers.
Therefore, the protocol satisfies $k$-resiliency for solution preference.

\subsubsection{Data Transfer}
A set $S$ of providers inputs a value from a domain $D$.
Providers from a set $O$ either output a value from $D$ or $\bot$.
When all providers in $S$ have the same input,
we want them to output the same value in $D$ when they follow the protocol.
We only require an implementation to be $k$-resilient if $|S|,|O| > k$,
since otherwise coalitions can always manipulate the output of this block.

\begin{property}
\label{prop:data-transfer}
Given that $|S|,|O| > k$ and all providers have the same input $v$,
an implementation $P$ of the data transfer must satisfy two conditions: 
(1) if all providers follow $P$, then they eventually output $v$;
and (2) $k$-resiliency for solution preference.
\end{property}

We propose a simple $k$-resilient implementation of this block,
where providers in $S$ broadcast their input to all providers in $O$.
In the end, if some provider $j \in O$ detects two different values,
then $j$ outputs $\bot$. Given that all providers
have input $v$ and that $|S|,|O| > k$,
they eventually output $v$, and
no coalition $K$ of up to $k$ providers
can cause all providers to output $v' \notin \{v,\bot\}$.
By solution preference, no provider in $K$ gains if someone lies
about the input $v$ or omits a message.

\subsubsection{Analysis}
Theorem~\ref{theorem:allocator} shows that every implementation
of the above framework satisfies the four conditions of Property~\ref{prop:allocator}.

\begin{theorem}
\label{theorem:allocator}
Every protocol $P$ that implements the parallel allocator
satisfies Property~\ref{prop:allocator}.
\end{theorem}

%\subsection{Proof of Theorem~\ref{theorem:allocator}}
\begin{proof}
\label{app:allocator}
We show that $P$ ensures (1) correct simulation of $\A$; (2) resilience to collusive influence;
(3) input validation; and (4) $k$-resiliency for solution preference if all providers have the same input.
First, we show (1). Suppose that all providers input the same vector $\vec{b}$
and follow $P$. We show that every provider outputs the same pair $(x,\vec{p})$
with probability $\A(x,\vec{p} \mid \vec{b})$. We show using induction that,
if the decomposition of $\A$ into tasks is done correctly and we fix all random
numbers, then at every task $T$ every provider $j$ that executes $T$ has the same output
that she would have if $j$ executed $\A$ locally with the same random numbers.
This is true for the first task by (2) of Property~\ref{prop:iv}. In the inductive step,
the input at each task depends only on the output of a set of tasks.
For each of those tasks $T$, by the induction hypothesis,
a set $S$ of at least $k+1$ providers computes the 
same result and inputs it to the data transfer;
by (1) of Property~\ref{prop:data-transfer}, all providers that execute $T$
receive that value and perform the same computation as they would if they were executing $\A$.
This implies that all providers output the same pair at the end.
By (1) of Property~\ref{prop:common-coin}, at every invocation
of the common coin, all providers input the same distribution $\Pi$ and output the same random number
distributed according to $\Pi$, where $\Pi$ is specified by $\A$. This proves (1).

Now, we show (2). Fix a coalition $K$ and suppose that all providers 
not in $K$ follow $P$ with input $\vec{b}$ and providers in $K$ follow an arbitrary $P' \ne P$.
The only way that providers in $K$ could cause providers not in $K$
to return pair $(x,\vec{p})$ with probability higher than $\A(x,\vec{p} \mid \vec{b})$
is if the result of some task used in the input of another task or as the final output
is not distributed as specified by $\A$ and $\vec{b}$.
Since each task is executed by more than $k$ providers, 
using an identical reasoning to the proof of (1), we can show using induction
that providers in $K$ cannot manipulate the probability distribution over the results
of each task, except only by increasing the probability of some provider 
not in $K$ outputting $\bot$. Here, we use the fact that,
by (3) of Property~\ref{prop:iv} and
(2) of Properties~\ref{prop:common-coin},~\ref{prop:data-transfer},
providers in $K$ cannot manipulate the probability distribution over
outputs of the building blocks in a way that increases
the expected utility of some provider in $K$. This proves (2).

Condition (3) follows by (2) of Property~\ref{prop:iv}.
To show (4), we first need to prove that providers have preference for a solution at all invocations
of building blocks, assuming that they have preference for a solution of the allocator.
Fix a coalition $K$. It is clear that providers in $K$ prefer that providers not in $K$
do not output $\bot$ at all invocations. Now, we use backward induction to show
that they prefer that providers not in $K$ never return different values.
In the last invocation, this is clearly true by preference for a solution of the allocator.
Continuing backwards, if two providers not in $K$ output different values
at the same invocation of some block, then either they output different pairs at the end or 
input different values at the following invocation of the data transfer,
which by the hypothesis is never preferable to outputting the same value at the considered invocation.
By the proof of (2), providers in $K$ cannot manipulate the final outcome
by not outputting the same values at all invocations, so
they also prefer to output the same values as providers not in $K$,
showing solution preference at all invocations. This also shows that
providers prefer to have the same input at all invocations.
Thus, given that all providers have the same input,
no provider in $K$ can increase its expected utility
if some provider $j \in K$ does not compute each task correctly.
By (3) of Property~\ref{prop:iv} and 
(2) of Properties~\ref{prop:common-coin} and~\ref{prop:data-transfer},
$P$ is a $k$-resilient equilibrium.
\end{proof}

\subsection{Resource Allocation Instances}
\label{sec__dist_auctioneer_instances}

We now show how our framework can be applied to two different bandwidth allocation problems in the context of community networks.
For that purpose, we resort to two different 
algorithms that have been proposed in the literature 
to solve bandwidth allocation for users in providers. 
These algorithms rely on standard and double auctions respectively, 
and have different computational properties: 
the double auction algorithm 
provides an example of a graph with only one task that is not computationally intensive, 
such that decomposing its execution into 
parallel tasks does not provide a performance gain; 
the standard auction algorithm provides a graph with multiple computationally 
intensive tasks that can be parallelised. 
Later in the chapter (\Cref{sec__dist_auctioneer_evaluation}), we will use these examples to evaluate 
the performance of implementations of the framework. 
We use the double auctions example to measure a worst-case overhead of executing all building blocks of the framework 
compared to an execution with a centralised trusted auctioneer, 
and we use the standard auctions example to show that the improvements of
parallelisation can outweigh the added overhead when the execution time is dominated by computation.

\subsubsection{Double Auction}
\label{sec:instances-double-auction}
Consider an auction where each provider has a limited bandwidth to be allocated to multiple users,
and each user has a demand of bandwidth that may be satisfied by multiple providers.
Both the users and the providers declare in their bids 
the value given to a unit of allocated bandwidth. 
An allocation gives the amount of bandwidth for each user
allocated in each provider. We want to ensure truthfulness in expectation and budget balance.
For this purpose, we use the algorithm $\A$ of~\cite{Zheng2014Star},
which provides the above properties at the expense of social welfare. 
The idea is to order the providers by increasing
value and to order the users by decreasing value.
Then, users are allocated by their order to the providers using the water-filling method:
the maximum amount of bandwidth of each user is allocated
to the first available provider without exceeding its capacity,
and any unsatisfied demand of that user is allocated to the following providers using the same method.
Since the most computationally intensive task of this algorithm
is sorting, in most practical settings there is no performance gain in parallelising the execution of $\A$.
Instead, every provider executes $\A$ locally and outputs the result.
Hence, we never need to invoke the data transfer building block.


\subsubsection{Standard Auction}
\label{sec:instances-vcg}
Consider a variation of the double auction where providers
do not send bids and each bidder can only have its bandwidth demand
allocated in a single provider. Here, we aim for truthfulness
in expectation, maximal social welfare, and computational efficiency.
It is well known that a VCG mechanism can be used to provide the first two guarantees.
The difficulty is that determining the maximal
social welfare is in general an NP Hard problem,
which conflicts with the goal of computational efficiency.
To address this issue, we use the algorithm of~\cite{Zhang2015Truthful}
which adapts the VCG mechanism to achieve a tradeoff
between the two conflicting requirements. Specifically,
\cite{Zhang2015Truthful}~offers a $(1-\epsilon)$ approximation
of maximal social welfare for an arbitrarily small $\epsilon$,
while terminating in polynomial time according to smoothed analysis.

Interestingly, the randomised algorithm proposed
in~\cite{Zhang2015Truthful} has the potential for parallelisation.
In a course manner, the algorithm can be divided into three steps, depicted
in Algorithm\,\ref{alg:bandwidth-allocation-randomised}. The first
step derives an approximately optimal allocation of users to providers.
This step is hard to parallelise effectively in a distributed system,
so we run it in a single sequential task.
The second step calculates the payments for each user based on the result of the first step.
This step is computationally intensive and the payments for each user can be computed
independently and, therefore, can be easily parallelised.
The final step gathers all intermediate results to produce the output.
In our implementation, the first and third steps are executed by all providers.
In the second step, we group the providers into $c$ groups, each containing at least $k+1$
providers. Each group is assigned the computation of the payments
of a subset of $n/c$ users. Then, all providers of a group
execute the data transfer block to transfer the resulting payments to all providers.

\begin{algorithm}[tbp]
	\caption{Standard auction allocator}
	\label{alg:bandwidth-allocation-randomised}
	\small

	\begin{algorithmic}[1]
		\State Task 1: Calculate the allocation solution $x$
		\For {Each subset $S$ of bidders in parallel}
			\State Task 2.$S$: calculate payment $p_j$ of every $j \in S$
		\EndFor
		\State Task 3: Collect the outputs of each task with the data transfer and output $(x,\vec{p})$
	\end{algorithmic}

\end{algorithm}

 

% *** Section: Evaluation/Experiments


% Section: EVALUATION
\subsubsection{Experiment Setup}
\label{sec:prototype-evaluation}

We deploy the prototype of the regulation component of the cloud coordinator from community cloud management system
in the Community-Lab testbed, which is developed by the CONFINE European project~\cite{Braem2013}. 
The cloud coordinator components are installed on nodes of the Community-Lab testbed, 
which consist of Jetway JBC372F36W devices, and are equipped with an Intel Atom N2600 CPU, 4GB of RAM and 120GB SSD. % http://www.jetwaycomputer.com/JBC372F36.html
Depending on the experiment, one or two nodes operate as SNs, while each ON hosts between one and four VM instances. 
Note, however, that since OpenWRT has limited supported for either containers-based or full virtualization, 
in these experiments the nodes submit and process the requests but VMs are not actually created on the ONs.
The objectives of the experiments are twofold: 

\begin{enumerate}
    \item Experiment 1: Assess the prototype operation regarding the incentive-based resource assignment algorithm in a local community cloud scenario.
    \item Experiment 2: Study the coordination between SNs from different zones in the federated community cloud scenario with heterogeneous resource distribution.
\end{enumerate}


%%%%%%%%%%%%%%%%%%%%%%%%%%%%%%%%%%%%%%%%
%%%%%%%%%%%%%%%%%%%%%%%%%%%%%%%%%%%%%%%%
\subsubsection{Resource Assignment in Local Community Cloud Scenario}
\label{sec:resource-assign-local}

In order to study the performance of the prototype in a real deployment of a local community cloud, 
we install our software components in four ONs and one SN in Community-Lab testbed, 
which are connected to the Guifi.net community network. 
Each node behaves as an ON but with different configuration, in order to have a heterogeneous set of cloud resources. 
The four nodes include
f101 sharing 1 out of total 2 VMs, 
f102 sharing 3 out of total 3 VMs, 
f103 sharing 1 out of total 3 VMs, and
f104 sharing 1 out of total 1 VM.
Each ON sends request for VM instances to the SN at regular intervals. 
VMs are requested for 20 seconds interval at a time. 
Each ON requests as many VMs as its total capacity, for example node f101 always requests 2 VMs.
If the request is accepted by the SN, the ON obtains the VMs for the next 20 seconds. 
If the request is rejected, the ON waits for 5 seconds before making any further requests.
The experiment is run with this setup for around 5 minutes. 
We analyse the different aspects of the system behaviour in the following.


%%%%%%%%%%%%%%%%%%%%%%%%%%%%%%%%%%%%%%%%
\subsubsection{Resource Utilisation}

Figure~\ref{fig:resource-util-graphs} shows the level of resource utilisation in the system in terms of the number of reserved VMs versus the total number of VMs.
It can be seen that resource utilisation varies widely and 100\% utilisation, meaning all the VMs being occupied, occurs only for short intervals.
This is because as nodes obtain VMs they spend their credit, 
and if they are not able to earn credit by contributing VMs, 
their credit gets below a certain threshold, and they can no longer request more VMs.
At approximately 80th second, the utilisation gets very low.
Nodes then need to earn credit by providing VMs to others before they can request VMs again.
So even though VMs are available, they cannot be utilised due to the lack of credit in the system.

%% FIGURE
\begin{figure}[tbp]
	\centering
		\includegraphics[width=0.65\textwidth,keepaspectratio]{graph_resource_util}
		\caption{Overall resource utilisation of the four ONs}
		\label{fig:resource-util-graphs}
\end{figure}


%%%%%%%%%%%%%%%%%%%%%%%%%%%%%%%%%%%%%%%%
\subsubsection{Credit Distribution}

Figure~\ref{fig:credit-distrib-graphs} shows the credit distribution among the four ONs during the 5 minutes of the experiment.
A node's credit is affected by how many VMs it shares and how much credit it spends to obtain VMs.
When a node shares most of its capacity, like ON f102 providing all its 3 VMs, 
it earns more credit and so maintains a high credit level during the experiment.
On the other hand, when a node continuously consumes VMs like ON f101 and f104, 
it keeps on spending any credit that it earns, so its credit does not increase beyond a certain level. 
Of particular interest is the behaviour of ON f103, which earns credit in the start and gets a spike in credit level halfway through the experiment, but then quickly spends it as it requests VMs from others.

Note that an ON's credit can be negative or higher than 100\% of the total credit 
because in the current implementation SN can allow requests from ONs with zero or or less than zero credit up to some extent. 
The ONs with zero or negative credit can, of course, continue to provide VMs and earn credit, they can only not request VMs if their credit is negative and below a certain threshold.
This allows the nodes without any credit an opportunity to continue participating in the system and increase their credit by contributing resources. 

%% FIGURE
\begin{figure}[tbp]
	\centering
		\includegraphics[width=0.65\textwidth,keepaspectratio]{graph_credit_distrib}
		\caption{Distribution of credit among the four ONs}
		\label{fig:credit-distrib-graphs}
\end{figure}


%%%%%%%%%%%%%%%%%%%%%%%%%%%%%%%%%%%%%%%%
\subsubsection{Success Ratio}

Figure~\ref{fig:success-ratio-graphs} shows the ratio of the fulfilled requests for each node, which is affected by the level of credit of the node and the amount of resources available in the system.
ON f104 has the most success since it requests only one VM at a time while ON f103 has the least success since it requests 3 VMs, which is half of the total shared VMs in the system. 
ON f101, on the other hand, gets its requests rejected because of the lack of credit. 
Therefore, this node has to wait to gain the needed credit.


%% FIGURE
\begin{figure}[tbp]
	\centering
	\includegraphics[width=0.65\textwidth,keepaspectratio]{graph_success_ratio}
	\caption{Ratio of fulfilled and rejected requests}
	\label{fig:success-ratio-graphs}
\end{figure}


%%%%%%%%%%%%%%%%%%%%%%%%%%%%%%%%%%%%%%%%
%%%%%%%%%%%%%%%%%%%%%%%%%%%%%%%%%%%%%%%%
\subsubsection{Resource Assignment in Federated Community Cloud Scenario}
\label{sec:resource-assign-federated}

In this experiment, we set up two local clouds, each with one SN and four ONs to study the federated community cloud scenario, as illustrated in Figure~\ref{fig:federated_cloud}.  
Table~\ref{tab:resource-distribution} shows the two cases with different number of VMs available in the two zones.
In the case of scarce capacity (case 1), the nodes in the SN1 zone share very few VMs compared to nodes in SN2 zone.
In the case of equal capacity (case 2), the nodes in both the zones share the same number of VMs.

%% TABLE
\begin{table}[tbp]
    \renewcommand{\arraystretch}{1.3}
   	\footnotesize    
    \caption{Two cases with different resource distribution between zones}
    \label{tab:resource-distribution}
    \centering

    \begin{tabular}{@{} l  c  c  c  c  c @{}}
    \hline
    \multicolumn{2}{c}{} 	& \multicolumn{2}{c}{Case 1: Scarce Capacity} & \multicolumn{2}{c}{Case 2: Equal Capacity}  \\ \hline
    SNs 					&       ONs     &   Total VMs   &   Shared VMs  &   Total VMs   &   Shared VMs \\ \hline
    \multirow{4}{*}{SN1}    &       ON1     &   3     	    &	1			&    3	        &	2   \\
                            &       ON2 	&	3	        &	1			&    3	        &	3   \\
                            &       ON3 	&	3	        &	1			&    3	        &	2   \\
                            &       ON4 	&	1	        &	1			&    1	        &	1   \\
    \hline
    \multirow{4}{*}{SN2}    &       ON1     &   3     	    &	2			&    3	        &	2   \\
                            &       ON2 	&	3	        &	3			&    3	        &	3   \\
                            &       ON3 	&	3	        &	2			&    3	        &	2   \\
                            &       ON4 	&	1	        &	1			&    1	        &	1   \\
    \hline
    \end{tabular}
\end{table}     

Figure~\ref{fig:multiple-snzone-graphs} shows the proportion of the requests fulfilled by VMs provided by the other zone.
With scarce capacity in SN1 zone, around 50\% of the requests are fulfilled by VMs provided by SN2 zone.
SN2 with sufficient capacity is able to meet most of the requests from VMs within the same zone, forwarding less than 15\% requests to the other zone.
In the second case, when both zones have the same available capacity, most of the requests get processed within the same zone for both the SNs.
This shows that a federated community cloud scenario extends the resources assigned to zones with limited capacity.

%% FIGURE
\begin{figure}[tbp]
	\centering
	\includegraphics[width=0.55\textwidth,keepaspectratio]{graph_multiple_snzone}
	\caption{Resources assigned from different SN zones}
	\label{fig:multiple-snzone-graphs}
\end{figure} 

% *** Section: Related Work


% Section: RELATED WORK
\section{Related Work}
\label{sec:related-work}

After the prevalence of public clouds~\cite{Armbrust2009}, there is now increasing interest in providing cloud services by harvesting excess resources from the idle machines connected to the Internet~\cite{Marinos2009}. 
Having different service level requirements and conditions, different solutions for how resources are contributed to build clouds have been found. Commercial clouds have dedicated resources that are financed by the users who pay in hard currency to use the cloud services.
Previous distributed multi-owned computing platforms like Seti@Home~\cite{Anderson2002}, HTCondor~\cite{Thain2005} and Seattle~\cite{Cappos2009} have relied on altruistic contribution of volunteer users. 
PlanetLab~\cite{Chun2003} requires for granting resource usage a prior fixed contribution before the services are made available.
None of these cases, however, correspond to the concrete situation of community networks.
In order to build a cloud platform within a community network, there is a need to create incentives to encourage active participation from the members of the community. 

Various incentive mechanisms have been studied for P2P and decentralised systems that address different requirements for ensuring a sustainable volunteer-based system~\cite{Zhang2010}.
P2P systems like BitTorrent~\cite{Cohen2003} incentivise using reciprocity based schemes where users consume resources in proportion to their contribution.
Most of these schemes do not take heterogeneity and varying capacity of different nodes into account so nodes with limited capacity are at a disadvantage because they do not benefit as much from the system even though they may be actively contributing to the system.
Recent work in cloud systems have also employed similar reciprocity based schemes, for example, Cloud@Home project~\cite{Distefano2012} envisages ensuring Quality of Service (QoS) using a rewards and credit system.
Fixed contribution schemes~\cite{Chun2003} need centralised management which are not suitable and scalable for decentralised systems like community networks.
Monetary based schemes~\cite{Punceva2013, Roovers2011, Petri2010, Toka2007} are founded on economic models and need careful micromanagement which makes it complicated to implement for a large decentralised system like community networks.

Regarding the different incentive schemes, our approach takes advantage of elements of the monetary-payment scheme, in the sense that credits are used to reflect the interchange of resources between consumers and providers. 
These credits are part of the components of the incentive mechanism that we propose for community clouds. 
We notice that none of the found related work focus on wireless community networks such as targeted by us.



% *** Section: Future Work
%
%
%\section{Discussion}
%\label{sec__discussion}

% *** Section: Conclusion


% Section: CONCLUSION
\section{Conclusion}
\label{sec:conclusion}

Community clouds are motivated by the additional value they would bring to community networks. 
Deploying applications in community clouds will boost the usage and spread of the community network model as ICT infrastructure for society. 
This paper builds upon the topology of community networks to derive two community cloud scenarios, local community cloud and federated community cloud. 
A community cloud architecture is then proposed which fits into these scenarios. 
The need for an incentive mechanism in order to community clouds to happen is stated, since for the contribution of any resources the motivation of the users is needed. 
This incentive mechanism is specified and implemented in a simulator in order to be able to perform assessments for large scale scenarios. 
With simulation experiments we characterised the behaviour of different settings of the incentive mechanism and evaluated the success ratio of nodes and resource utilisation. 
A deeper analysis of the behaviour allowed us to better understand the influence of the different configuration options. 
The incentive mechanism has been designed and evaluated taking into account the conditions of community networks. 
Therefore, we expect our results to be transferable to a prototype of a real community cloud system.
 
 

% use section* for acknowledgement
\section*{Acknowledgement}
This work was supported by the European Community Framework Programme~7 FIRE Initiative projects Community Networks Testbed for the Future Internet (CONFINE), FP7-288535, and CLOMMUNITY, FP7-317879. 
Support is also provided by the Universitat Polit\`ecnica de Catalunya BarcelonaTECH and the Spanish Government through the Delfin project TIN2010-20140-C03-01.


%
% ---- Bibliography ----
%
\bibliographystyle{LNCSTemplate/splncs}
\bibliography{Bibliography/references}
%


%\clearpage
%\addtocmark[2]{Author Index} % additional numbered TOC entry
%\renewcommand{\indexname}{Author Index}
%\printindex
%\clearpage
%\addtocmark[2]{Subject Index} % additional numbered TOC entry
%\markboth{Subject Index}{Subject Index}
%\renewcommand{\indexname}{Subject Index}
%\input{subjidx.ind}

\end{document}
