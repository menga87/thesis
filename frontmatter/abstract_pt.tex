% !TeX spellcheck = pt_PT

% Abstract in Portuguese

\chapter*{Resumo}

A Internet e as tecnologias de comunicação têm reduzido os custos para comunidades colaborarem, 
levando a novos serviços como conteúdo gerado pelos utilizadores e computação social, 
surgindo também, através de colaboração, infraestruturas construídas colectivamente, tais como redes comunitárias.
Enquanto as redes comunitárias focam-se unicamente na partilha de largura de banda, 
as núvens comunitárias alargam esta partilha para providenciar aplicações de interesse local, 
implementadas dentro de redes comunitárias através de esforços colaborativos para providenciar infraestruturas em núvem.
As núvens comunitárias complementam os tradicionais fornecedores de núvens públicas de larga escala, de forma similar ao modelo de núvens de limite, 
trazendo tanto conteúdo como computação para mais perto dos utilizadores nos limites da rede.
As núvens comunitárias são baseadas no princípio de partilha recíproca e a maioria dos seus utilizadores são movidos por princípios altruístas.
Contudo, tal como qualquer outra organização humana, estas redes não são imunes à sobreutilização, parasitismo, 
ou sub-provisão, especialmente em situações onde os utilizadores possam ter motivações para competir por recursos escassos.
Nesta tese focamo-nos nos mecanismos de base para incentivo de regulação de recursos, 
para derivar formas práticas de implementar arbitragem quando ocorre disputa por recursos limitados.
Primeiro projectamos estes mecanismos de regulação ao nível local, 
onde os laços sociais entre membros da comunidade são mais fortes e implicam confiança, e garantem adesão às políticas do sistema.
Em seguida alargamos os mecanismos para comunidades de utilizadores não-confiáveis maiores, 
onde utilizadores racionais podem estar motivados a desviar-se do comportamento esperado para ganho pessoal, 
e desenvolvemos uma estrutura distribuída para garantir confiança na regulação de recursos.
Tais mecanismos incentivam à contribuição dos membros da comunidade, e ajudando no sentido da adopção, 
sustentabilidade e crescimento do modelo de núvens comunitárias.

\vfill

\subsection*{Palavras Chave}
	nuvem comunitária;
	redes comunitárias;
	computação em nuvem;
	mecanismos económicos


