%!TEX root = ../dissertation.tex
% the abstract

Internet and communication technologies have lowered the costs for communities to collaborate, leading to new services like user-generated content and social computing, and through collaboration, collectively built infrastructures like community networks have also emerged. 
While community networks focus solely on sharing of network bandwidth,
community network clouds extend this sharing to provide for applications of local interest deployed within community networks through collaborative efforts to provision cloud infrastructures.
Community network clouds complement the traditional large-scale public cloud providers similar to the model of decentralised edge clouds by bringing both content and computation closer to the users at the edges of the network. 
Community network clouds are based on the principle of reciprocal sharing and most of their users are moved by altruistic principles. 
However, as any other human organisation, these networks are not immune to overuse, free-riding, or under-provisioning, specially in scenarios where users may have motivations to compete for scarce resources.
We focus in this thesis on the incentives based resource regulations mechanisms to derive practical ways of implementing arbitration when such contention for limited resources occurs.
We first design these regulation mechanisms for the local level where stronger social relationships between the community members imply trust, and ensure adherence to the system policies.
We next extend the mechanisms for larger communities of untrusted users, where rational users may be motivated to deviate for their personal gains, and develop a distributed framework for guaranteeing trust in the resource regulation.
Such mechanisms assist in encouraging contribution by the community members, and will help towards adoption, sustainability, and growth of the community cloud model.

\vfill

\subsection*{Keywords} 
		community cloud;
		community networks; 
		cloud computing; 
		economic mechanisms
