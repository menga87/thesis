% !TeX spellcheck = es_ES

% Abstract in Spanish

\chapter*{Resumen}

El Internet y las tecnologías de la comunicación han bajado los costos de colaborar en comunidad, 
dando lugar a nuevos servicios, como los contenidos generados por usuarios y la informática social y, 
por medio de la colaboración, han surgido infraestructuras construídas colectivamente, como las redes comunitarias. 
Mientras las redes comunitarias se centran exclusivamente en el intercambio de ancho de banda de la red, 
las nubes comunitarias extienden este intercambio para proporcionar aplicaciones de interés local, 
desplegadas en las redes comunitarias a través de actividades de colaboración para proveer infraestructuras en la nube. 
Las nubes comunitarias complementan a los proveedores tradicionales de la nube a gran escala, 
en un modo similar al modelo de las nubes descentralizadas, 
trayendo tanto el contenido como la computación más cerca hacia los usuarios en los extremos de la red. 
Las nubes comunitarias se basan en el principio de compartir recíprocamente y la mayoría de sus usuarios son movidos por principios altruistas. 
Sin embargo, como cualquier otra organización humana, estas redes no son inmunes al uso excesivo, 
al parasitismo, o al bajo-aprovisionamiento, especialmente en escenarios donde los usuarios pueden estar motivados a competir por recursos escasos. 
Nos centramos en esta tesis en los mecanismos de regulación de recursos basados en incentivos
para derivar formas de aplicación práctica del arbitraje cuando se produce tal contención por recursos limitados.
Primero diseñamos estos mecanismos de regulación a nivel local, 
donde las fuertes relaciones sociales entre los miembros de la comunidad generan confianza 
y aseguran la adhesión a las políticas del sistema.
A continuación, extendemos los mecanismos para comunidades más grandes de usuarios no confiables, 
donde usuarios racionales pueden ser motivados a desviarse por sus ganancias personales, 
y desarrollamos un marco distribuido para garantizar confianza en la regulación de recursos. 
Tales mecanismos ayudan a fomentar la contribución de los miembros de la comunidad, 
y ayudan a la adopción, la sostenibilidad y el crecimiento del modelo de nube comunitaria.

\vfill

\subsection*{Palabras Clave} 
		nube comunitaria;
		redes comunitarias;
		computación en la nube;
		mecanismos económicos

