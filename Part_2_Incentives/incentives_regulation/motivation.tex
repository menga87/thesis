
\section{Motivations}
\label{sec__incentvives_motivation}

Community networks are based on the principle of reciprocal sharing, 
which is formalised by the agreements users accept when they join the network.
The Wireless Commons License~\cite{WCL2010} or Pico Peering Agreement~\cite{PicoPeering2005} 
is adopted by many community networks to regulate network sharing. 
The underlying principle is that users allow all the traffic to transit over the links that they own,
and this is in return for the access they enjoy to the community network.
The agreements are enforced, in most cases, through the social context.
For example, members of Guifi.net have mailing lists and many regularly hold meetings where any complaints 
and issues are also addressed~\cite{Baig2015Guifi}.
In the worst situation, users may be excluded from using the services, if their behaviour continues to
be damaging to the operation of the community network.

Community networks in most cases tend to avoid having sophisticated mechanisms in place like auditing, billing, 
or other advanced pricing or auction-based mechanism, 
since they find it not in line with the sharing and open spirit of the community networks. %TODO better word for Sharing
Moreover, such mechanisms introduce additional overheads to the operation of the network,
and in many cases may be difficult for users to understand and may discourage participation.
When problems do occur, these are often addressed in ad-hoc manner and at the local level, 
for example, the network administrator may shut down some of the links.
For persistent problems, the community often chooses to increase the capacity by buying more equipment and adding new links.
For instance, in Guifi.net, some nodes have been successfully crowd-funded~\cite{Baig2015Guifi} if such a node was needed by several people. 
Crowd-funding of a node happened when for a group of people an infrastructure improvement was necessary. 
For example, an isolated zone of Guifi.net established a super node to connect to other zones. 
%In such a case, the node has been purchased with the contributions of many people. 
%The location of such a node follows strategic considerations, 
%trying to optimise the positive effects on the performance 
%that are achieved with the addition of the new infrastructure. 

We think that such a situation may not be sufficient for sustaining a community cloud ecosystem.
This is because the contribution required from the users of community cloud, both in terms of the equipment, 
capital investment, and maintenance costs, and their effort, knowledge, and time,
is orders of magnitude larger than what is needed to keep the community networks running.
In the absence of any resource regulation mechanism, users will lack incentives to contribute resources.
When all the computing and storage are made available at no cost to the community, 
it would be difficult to find many users willing to contribute solely for altruistic reasons,
and even when the resources are made available,
they will get consumed quickly by the free-riding users.
Therefore, we propose that incentives based resource regulation has to play a key part
in the sustainability of a community cloud ecosystem.
