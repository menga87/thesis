
%%%% This text is currently copied from EMJD PhD Report, June 2015. %%%%

Recent developments in communication technologies like the Internet, email and social networking have significantly removed the barriers for communication and coordination for small to large groups, 
bringing down the costs that obstructed collaborative production before the era of the Internet.
The ICT revolution ushered in group communication and collaborative production with popular applications now widely adopted, like social networking, social bookmarking, user-generated content, photo sharing, and many more. 
Even infrastructures based on a cooperative model have been built, for example community wireless mesh networks~\cite{Braem2013} gained momentum in the early 2000s in response to limited options for network connectivity in rural and urban communities.

Community networks represent a social collective to build ICT infrastructures for serving interests of the rural and urban communities.
Volunteers in their local communities use off-the-shelf network equipment and open unlicensed wireless spectrum to provide network and communication services.
These community networks have been popular, and have recently also employed fibre optic links~\cite{Baig2015Guifi},
for example
	Athens Wireless Metropolitan Network (AWMN)~\cite{Awmn}, 
	Freifunk~\cite{Freifunk},
	FunkFeuer~\cite{Funkfeuer},
	Guifi.net~\cite{Guifinet}, and
	Ninux~\cite{Ninux},
are some of the networks deployed in Europe, having up to 28,000 nodes~\cite{Baig2015Guifi}.
Community networking thus presents an emerging model for the Future Internet across Europe and beyond,
allowing for communities of citizens to build, operate and own open IP-based networks, 
and enabling individual and collective digital participation. 


Community networks are based on the concept of reciprocal sharing, 
but this sharing is limited to network bandwidth, 
and does not extend to other computing resources.
Community clouds aim to address this limitation, enabling the sharing of all types of computing resources, 
following the model of cloud computing, and assist in developing services and applications of local interest within community networks. 
Community cloud in this context refers to the cloud hosted on community-owned computing and communication resources providing services of local interest. 


Community clouds, similar to community networks, are based on volunteer efforts, 
so need to provide tangible or intangible benefits to the users in order to keep them engaged.
This requires that resource allocation mechanisms are designed 
to incentivise contribution from the users,
so as to enable the community cloud transition from inception through early adoption to finally ubiquitous usage, 
leading to a sustainable and viable ecosystem.
