

% Section: RELATED WORK
\section{Incentives}
\label{sec__incentives_related_work}

Many volunteer and distributed platforms appeal to users' altruistic instincts. 
Projects like 
	SETI@Home~\cite{Anderson2002}, 
	BOINC~\cite{Anderson2004}, and
	Folding@home~\cite{Beberg2009}
propose to solve challenging scientific problems, which encourage users to contribute the idle resources of their machines.
Other projects like PlanetLab~\cite{Chun2003} and various Grid systems~\cite{Foster2003} require
some mutually-agreed upon level of contribution as a pre-condition for participating in the system.

Other earlier popular peer-to-peer (P2P) file sharing programs like Kazaa, Gnnutella, Napsters, and eMule among others~\cite{Rodrigues2010}, 
allowed users to upload and download files for free, % as in \emph{gratis}, 
and often suffered from the issues like free riding, under-provisioning, etc. 
This has led to designing incentive mechanisms in P2P systems to ensure that users actively contribute. 
For example, BitTorrent, using reciprocity principle, only allows users to download content
if they also upload part of it to other BitTorrent users. 
Similar incentive mechanisms have been extensively studied for other P2P systems~\cite{Babaioff2007, Shen2010, Zhang2010}.

Community networks are also based on the principle of reciprocal sharing, 
and offer various tangible and intangible benefits to their users.
Bina and Giaglis~\cite{Bina2006} have explored various psychological and 
social motivations of the users of AWMN community network. 
Reciprocal resource sharing is, in fact, part of the membership rules 
or peering agreements~\cite{PicoPeering2005} of many community networks. 
The Wireless Commons License (WCL)~\cite{WCL2010} of many community networks states 
that the network participants that extend the network, e.g.~contribute new nodes, 
will extend the network in the same WCL terms and conditions, 
allowing traffic of other members to transit on their own network segments. 
Therefore, resource sharing in community networks from the equipment perspective 
refers in practice to the sharing of the nodes' bandwidth. 
This sharing, done in a reciprocal manner, enables the traffic from other nodes 
to be routed over the nodes of different node owners, and 
allows community networks to successfully operate as IP networks.

Most of these incentive mechanisms are based on the idea of reciprocating individual's \emph{contribution}, 
anybody who contributes more value to the system is allowed to reap more benefits from it.
However, this does not take into account that in many cases not all users are as rich in resources as others. 
Participatory Economics (Parecon) model envisions rewarding the users 
based on their effort, which is defined as contribution as a fraction of their capacity, 
instead of rewarding purely on the basis of their absolute contribution~\cite{Albert2004Parecon}. 
Effort-based incentives~\cite{Rahman2010, Vega2013Sharing, Vega2015Thesis} 
have been proposed based on Parecon principle to achieve fairness and 
improve social welfare, while better addressing the resource heterogeneity in the system. 
For instance, Rahman et al.~\cite{Rahman2010} apply effort-based incentives for file sharing in BitTorrent 
where some users with slow Internet connections cannot upload as much content as others. 
When deciding how much a user can download, they propose factoring in user's connection speed 
in addition to the data the user has uploaded.
Their results show that the effort-based mechanism remains incentive-compatible 
and improves efficiency and fairness as compared to purely contribution-based approach.
We also turn to effort-based incentives in \Cref{chap__trusted_auction} 
when devising mechanisms for resource sharing between the users of community network clouds.

