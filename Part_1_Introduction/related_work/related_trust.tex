
\section{Trust in Resource Allocation}
\label{sec__related_work_trust}

%\subsection{Trust Issue in Auctions}

The issue of trust in auctions is well-known and well-studied in economic theory,
for various type of auctions~\cite{Sandholm2000}.
In particular Vickrey auction, 
which forms basis of the celebrated VCG mechanism~\cite{Nisan2001}, 
is very susceptible to a lying auctioneer~\cite{Sandholm2000}.
VCG mechanism is often used in distributed systems and cloud computing
to ensure strategy-proofness in resource allocation, 
but it suffers from significant issues because of the trust 
required in the auctioneer~\cite{Sandholm2000}.
There are numerous distributed auction schemes 
proposed in literature~\cite{Guo2013, Zhou2008, Lai2004},
but the fact that they are decentralised in itself does not imply trust.
Without careful design, one or more agents participating in the
distributed auction can affect the results to their advantage.

The issue of an auctioneer cheating in second-price sealed-bid auctions, similar to VCG auctions,
has been so severe that fraud was commonplace in the stamps auctions 
of late 19th and 20th century~\cite{Lucking-Reiley2000},
which provide the first recorded example of using Vickrey auctions in practice.
More recently, such second-price auctions have been used by eBay~\cite{Lucking-Reiley2000Auctions} to sell goods, 
and by Yahoo, Google, and other Internet search companies to sell keywords-based 
online advertising~\cite{Edelman2007}.
It has been suggested that such auctions are viable for these Internet companies,
in the absence of any trust, because these companies have low commissions,
and conduct so many trades that it is not in their interest to cheat~\cite{Lucking-Reiley2000}.
But the issue of mistrust in the auctioneer has been prevalent, 
so much so that when Amazon introduced its
\enquote{spot instances} service using market-based pricing, 
there were doubts that Amazon was not using supply and demand to set the prices,
but instead employed a mathematical regression function to set the rates~\cite{BenYehuda2013}.

%\subsection{Implications of Lack of Trust}
This lack of trust can create various problems for resource allocation in decentralised systems, 
and we go through a few examples here.
InterCloud allows for multiple cloud providers to federate their resources 
to form a cloud market~\cite{Grozev2014}.
The current approaches mostly rely on bilateral agreements, and price negotiation.
However, as the number of providers increases this arrangement
will no longer be tenable.
The providers cannot put absolute trust in anyone among them
or in a third party to execute the auction fairly.
%
Social clouds~\cite{Caton2014} allow exchanging resources between users of online social networks.
Here, as well, the users have to trust the allocation and resource matching algorithm,
which even if it runs distributed on multiple machines,
cannot be trusted, since some of the users can \emph{tweak} the allocator to their advantage.
%
Secondary wireless spectrum markets~\cite{Zhou2008, Zhu2012} also apply auctions to ensure
strategy-proofness, but again fail if a trusted auctioneer is not available.
%
BitTorrent, like other similar popular P2P file sharing systems,
regulates access to available resources depending on the users' contribution.
However, users can easily use their BitTorrent clients to falsely report their contribution to be high.
To counter this, BitTorrent private communities have a central mediator
that dictates rules for uploading and downloading content, 
and tracks contribution and consumption by the users~\cite{Liu2010}.
However, the administrators and privileged users can affect the central
mediator to get unfair advantage.
%
Resource allocation in shared infrastructures like Grid systems also require incentive-compatible 
solutions~\cite{Lai2004, Khan2006Noncooperative, Buyya2002Economic}, %  Buyya2005, 
and many of the proposed approaches in the literature apply only 
when a trusted entity is available for executing the auction mechanisms.

A cheating auctioneer can pose a number of problems.
The major among them, that we focus on, is that the winning bids may not be selected fairly,
and the payment each winning bidder has to make is not calculated properly.
In open-bid auctions, like English and Dutch auctions~\cite{Krishna2009},
the above two issues are not a major problem as the bidding process is open to all the participants.
In sealed-bid first price auction, each winning bidder pays the amount she quoted,
so even though the auctioneer can cheat on selecting the winning bidders,
at least the winning bidders are sure they are paying the correct amount.
In sealed-bid second price auctions, the auctioneer can even cheat 
on the price the bidder has to pay, and this is a major issue~\cite{Sandholm2000}.

Other issues with auctions that we do not consider here include bidders cooperating
together to artificially lower the price, 
the providers making a coalition to keep the price high,
or the auctioneer learning from buyers' preference to increase the reserve bid price.
We do not focus on these and other issues, but in recent literature
cryptographic zero-knowledge based algorithms have been 
employed to tackle these problems~\cite{Micali2014, Lipmaa2003}.
Some of these solutions, because of their high computational complexity,
are better suited to infrequent auctions of highly valuable goods,
for example wireless spectrum auction by a government,
than for integrating in mechanisms for distributed systems
where the auctions are often repeated at short intervals.


\subsection{Allocation with Rational Users}
\label{sec__allocation_rational}

Most of the existing approaches, as we have discussed above, assume a single trusted entity 
or multiple trusted entities coordinating together
for executing the resource allocation mechanism. 
Unfortunately, this is an unreasonable assumption in many of today's fully
decentralised systems, where all nodes are either resource consumers,
resource providers, or both. 
In this case, there is no natural candidate that 
can be trusted by all other nodes to run the auction
algorithm faithfully, given that any node may extract some
benefit by perturbing the auction result. 
In some sense, all current distributed systems that 
rely on centralised auctions to perform resource
allocation are not fully decentralised, 
because they depend on a unique central point of control, 
which is the trusted node that runs the auction algorithm. 
Similarly, in the context of community network clouds, 
that comprise of untrusted service providers,
such a trusted entity in most cases is not feasible, 
and even if it existed, can impede the scalability.
This leads to the observation that there is
still a substantial gap that needs to be bridged to apply these
results in fully decentralised settings.
None of the above works, to the best of our knowledge, 
address the problem of resource allocation
in the absence of a trusted auctioneer.

%TODO Discuss Byzantine in more details??

The problem of simulating the behaviour of a trusted entity in an environment 
with only rational players has been approached in the literature of 
distributed systems~\cite{Aiyer2005,Halpern:04,Abraham:06,Abraham2013,Afek2014}.
%
Aiyer et al.~\cite{Aiyer2005} presented BAR model which involves Byzantine (or faulty) 
and rational players, as well as acquiescent players who follow the suggested protocols. 
They used BAR model to develop fault tolerant cooperative services using state machine replication.
%
% Distributed Computing Meets Game Theory: Robust Mechanisms for Rational Secret Sharing and Multiparty Computation
% Rational Secret Sharing and Multiparty Computation: Extended Abstract
In~\cite{Halpern:04,Abraham:06}, the authors addressed the
particular problems of secret sharing and multi-party computation
assuming the existence of a trusted mediator, 
and then studied conditions under which it is possible to simulate the mediator through a distributed protocol. 
They discussed $k$-resilient Nash equilibria
so that even if a whole coalition of up to $k$ players 
chooses joint strategies to defect, 
still no member of the coalition can increase its utility.
Their results apply even if there are only 2 players, 
so that multi-party computation can be performed with only two rational agents. 

%
Abraham et al.~\cite{Abraham2013} devised $k$-resilient equilibria solutions 
for the classic problem of electing a leader in an anonymous network in the presence of rational agents. 
They considered existence of $k$-resilient equilibria for several topologies 
like a unidirectional ring, bidirectional ring, or completely connected network, 
in both the synchronous and asynchronous case.
They showed that a fair ex post $k$-resilient equilibrium 
requires $n > 2k$, i.e.~the number of agents should be more than 
twice the size of the largest coalition for the system to reach equilibrium.
They also highlighted how involving cryptographic techniques 
can help achieve equilibrium, and result in more computationally efficient algorithms.

Afek et al.~\cite{Afek2014} proposed a building blocks approach 
for devising distributed $k$-resilient implementations 
to solve common distributed computing problems 
like leader election, consensus, and renaming. 
They extended the ideas from~\cite{Abraham2013} to develop building blocks 
that are all resilient in the presence of rational agents, 
and coalitions of rational agents to some extent as well.
They differentiated between different utility functions for distributed algorithms, 
e.g., utilities based on communication preference, 
solution preference, and output preference. 
Based on these preferences, they formulated two basic building blocks 
for game theoretic distributed algorithms, 
a wake-up building block and and a knowledge sharing building block,
that are resilient to these preferences. 
These blocks formed part of their solutions to 
leader election, consensus, and renaming problems.

None of these works devised distributed protocols for simulating the role of an auctioneer in an auction. 
We build on the work of~\cite{Abraham2013, Afek2014} to develop a distributed virtual auctioneer, 
which we focus on in \Cref{chap__trusted_auction}.
