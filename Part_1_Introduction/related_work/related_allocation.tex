

% Section: RELATED WORK
\section{Economic Based Resource Allocation}
\label{sec__resource_regulation}

% -------------- From ICDCS'16 Paper Submission -------------- %

Most distributed systems have limited resources that need to be shared by many nodes. 
For instance, in a network there is limited bandwidth that needs to be allocated to multiple nodes. 
In cloud applications, virtual machines (VMs) need to be allocated to different cloud users. 
Resource allocation is, therefore, a key problem in distributed systems, 
and there is a vast literature on resource allocation of shared resources, 
whether they be network bandwidth, or other general computing resources. %TODO Cite a survey or book

Distributed resource allocation is particularly challenging when nodes
operate under different spheres of control and may not be willing to cooperate. 
Namely, a resource allocation strategy that assumes that all nodes 
execute a given algorithm may break if nodes 
may extract benefits by deviating from the expected behaviour.
Many examples of this problem can be found in the literature. 
The works of~\cite{Lee2007} and~\cite{Xiao2013} illustrate how
a network user may attempt to monopolise the bandwidth utilisation 
if it has the opportunity. 
There is evidence that programmers can instrument their code 
to get an unfair advantage of several Unix schedulers~\cite{Grosu2005}. 
In shared infrastructures, like Grid systems, participating
users try to maximise their own usage to the detriment of 
the others~\cite{Lai2004}.
Dynamic wireless spectrum allocation suffers from unfair manipulation~\cite{Zhou2008}.
Social cloud computing~\cite{Caton2014}
and cooperative computing systems like BitTorrent~\cite{Liu2010}
suffer when users act selfishly in consuming resources.

An approach that has emerged as a viable alternative for the problem
above is to use economic models to address resource allocation, 
in particular by resorting to auction systems~\cite{Hurwicz1973, Riley1981}. 
As a result, an extensive literature exists on the use of several types of auctions 
to perform resource allocation in distributed systems~\cite{Nisan2001, Niyato2013, Waldspurger1992}. 
In particular, the advent of the cloud computing model, where many
clients may compete for the resources managed by one or more providers, 
has spurred the usage of different auction mechanisms for the cloud~\cite{Wang2012When}.

In these approaches, users are modelled as non-cooperative rational
players who are willing to pay for using resources or get paid for
providing those resources. Specifically, users declare to an
auctioneer the preference for different allocations of resources, and
the auctioneer executes some auction mechanism to derive an allocation
between users and resources that maximises social welfare (preferences of users for the allocation),
and the payments to be performed or received by each user. 
The aim is to obtain an allocation with a social welfare as close as
possible to the optimal while ensuring truthfulness from the
users, such that they do not have incentives to lie about their bids. 
In addition to maximal social welfare and truthfulness, other guarantees may be provided,
including computational efficiency and budget balance 
(the payments made by the users outweigh the payments received).

In the following, we look at the state-of-the-art in economic-based resource regulation,
first for community networks (\cref{sec__related_work_networks}), 
and then for cloud systems (\cref{sec__related_work_cloud}) 
both in the context of virtual machines %(\cref{sec__related_work_cloud_vms})
and network bandwidth. % (\cref{sec__related_work_cloud_bandwidth}) 
Next, we study managing resource sharing between federations of cloud providers (\cref{sec__related_work_cloud_fed}) .


\subsection{Community Networks}
\label{sec__related_work_networks}

Various pricing schemes, game theoretic mechanisms and auction-based approaches
for arbitrating network resources have been proposed and used in practice.
Maillé and Tuffin~\cite{Maille2014} provide a comprehensive survey of the historical
approaches and the current state-of-the-art from the viewpoint of telecommunication services.

Community networks comprise of wireless mesh and multi-hop networks,
and require cooperation among its users for proper functioning.
Game theory, both non-cooperative and cooperative, has been applied in literature for 
studying the incentives of its user and devising network allocation mechanisms.
%% Xiao2013: Fairness and efficiency tradeoffs for user cooperation in distributed wireless networks (INFOCOM)
Xiao et al.~\cite{Xiao2013} propose a general framework for studying the user cooperation in a network, 
and formalise the relationship between incentive, fairness and efficiency 
for cooperative networks.
% Xiao2013: Main focus is on Random Access (RA) networks. Uses cooperative theory. Introduces strong price of fairness idea (SPOF).
%
%% Zhou2014: Auction-based bandwidth allocation and scheduling in noncooperative wireless networks (ICC)
Zhou et al.~\cite{Zhou2014} develop a Vickrey-Clarke-Groves (VCG) based mechanism for non-cooperative users.
Since VCG mechanism cannot be directly applied because of the high computational complexity,
they modify it by using relaxation-based greedy algorithm 
in such a way that it still guarantees strategy-proofness and efficiency.
%
% Lee2007: Backbone construction in selfish wireless networks, SIGMETRICS
Lee et al.~\cite{Lee2007} focus on the problem of backbone construction with selfish users 
in a community network, who want others to relay their packets but want to avoid relaying packets for others. 
They propose an incentive-compatible protocol based on Volunteer's Timing Dilemma 
from non-cooperative game theory.
Their findings show that using their protocol the backbone forms quickly, 
with characteristics comparable to protocols designed only for altruistic users.

%TODO cite and explain works from secondary spectrum access ?

Community networks like Guifi.net, in practice, 
also put emphasis on social sharing agreements~\cite{PicoPeering2005, WCL2010}, 
and when conflicts occur enforce these agreements through social mechanisms~\cite{Baig2015Guifi}.
For our context of community network clouds, 
we accept the guarantees implied by the community networks, 
and assume that the network is owned by the whole community
so the traffic within the network between any two nodes is ensured to transit over intermediary nodes. 
Therefore, when we address the issue of bandwidth reservation in \Cref{chap__trusted_auction}, 
we focus solely on the gateways present in the community network that provide access to the Internet.


\subsection{Cloud Systems}
%\subsection{Auction-based Resource Allocation}
\label{sec__related_work_cloud}

With respect to the general computing resources, there
has been a lot of existing work in the context of 
Grid systems~\cite{Foster2003, Buyya2005, Courcoubetis2012}
and shared infrastructures like PlanetLab~\cite{Leon2013Economic}.
For instance, one of the system implemented for PlanetLab is Tycoon~\cite{Lai2004}, 
which is a distributed marked-based resource allocation system,
implementing proportional fairness using decentralised isolated auctions.
Tycoon allows users to differentiate their jobs based 
on their importance by specifying different bid amounts.
Auctioneers manage only local resources, 
and users submit separate bids to these auctioneers.
The bids remain valid until a user's credit gets low, 
so the mechanism reduces manual bidding overhead by the users. 
Resources are assigned in proportion to the bid amounts using a best-effort approach.
This allows Tycoon to achieve efficient usage of resources, while maintaining little allocation overhead.

Economic-based resource allocation mechanisms have been extensively explored for cloud 
computing~\cite{Niu2012, Popa2012, Shi2014, Wang2012When, Zhang2013Framework, Zhang2015Truthful, Zheng2014Star, Zheng2015}.
Amazon, one of the leading providers of public cloud services, was among the first to offer
cloud resources using market-driven prices~\cite{BenYehuda2013}.
Besides investigating how a provider can use economic-based mechanisms to maximise 
the utilisation of its resources and increase its revenue~\cite{Wang2012When}, 
the recent research has also looked into devising the best strategies for the users to
bid for the cloud resources in such a market~\cite{Zheng2015}.
Many works, for instance ~\cite{Zhang2014Dynamic, Zhang2015Truthful}, 
employ the celebrated Vickrey-Clarke-Groves (VCG) mechanism~\cite{Nisan2001}
for achieving truthfulness and a trade-off between 
maximal social welfare and computational efficiency.
The challenge in applying VCG mechanisms in clouds is that in most of these problems,
finding an optimal allocation is NP-hard,
and so traditional VCG mechanism cannot be applied.
One key line of work has been to use randomised algorithms~\cite{Zhang2014Dynamic, Zhang2015Truthful},
or linear programming decomposition techniques~\cite{Niu2013, Zhang2015Online},
to achieve strategy-proofness with a computationally feasible solution,
albeit achieving less than optimal social welfare.
The overall aim is to look for solutions with lower computational complexity, 
and higher social welfare, while ensuring strategy-proofness.

\subsubsection{Virtual Machines Allocation}
\label{sec__related_work_cloud_vms}

Cloud computing employs virtualization to package computing resources like CPU time, memory, and storage as virtual machines (VMs). 
Therefore, resource allocation in cloud, for the most part, deals with composing VMs 
from the hardware resources, and allocating them to the users in the most efficient manner. 
Cloud providers offer a variety of VM instances of different types, where type refers 
to the composition of different resources packaged in the VM instance. 
For example, a VM consisting of 2 virtual CPU units, 8 GB RAM and 100 GB storage is one type of VM. 
%% Zhang2015Truthful: A Truthful (1-ϵ)-Optimal Mechanism for On-demand Cloud Resource Provisioning
Zhang et al.~\cite{Zhang2015Truthful} extend this to allow users to request 
customised dynamically assembled VM types, 
bundling VMs from different geo-distributed data centres of the provider.
They use smoothed analysis and randomised reduction 
to design a randomised, highly efficient auction mechanism.
Their mechanism is general and expressive enough to encompass various cloud scenarios, 
and achieves truthfulness (in expectation), 
polynomial running time (in expectation), 
and $(1-\epsilon)$-optimal social welfare (in expectation) 
for resource allocation in a geo-distributed cloud, where $\epsilon \in (0,1)$.

%% Zhang2014Dynamic: Dynamic resource provisioning in cloud computing: A randomised auction approach
Zhang et al.~\cite{Zhang2014Dynamic} approach combinatorial
auctions of heterogeneous VMs by modelling social welfare maximization
as a mixed linear integer program. 
They design an efficient $\alpha$-approximation algorithm,
with $\alpha\! \sim\! 2.72$ in typical scenarios.
They use this algorithm as a building block for designing 
a randomised combinatorial auction that is computationally efficient, 
truthful in expectation, and guarantees the same social welfare approximation factor $\alpha$. 
They utilise a pair of tailored primal
and dual linear programs (LPs) to decompose fractional solution 
of social welfare maximization problem into a convex combination of integral solutions.
%The randomised auction achieves the same approximation ratio 
%in social welfare as the cooperative algorithm does.

%% Niu2013: An Efficient Distributed Algorithm for Resource Allocation in Large-Scale Coupled Systems

%TODO Any other recent work 

\subsubsection{Bandwidth Reservation}
\label{sec__related_work_cloud_bandwidth}

The focus remains mostly on efficiently allocating VMs in the cloud, 
even though the bandwidth, both upstream and downstream, to connect to VMs in the cloud is metered.
Bandwidth allocation and reservation gains significance, 
in particular when the applications are network-intensive or have real-time constraints.
Some prime examples are video-streaming, video-on-demand and cloud-based gaming. 
Recent work has explored various economic based bandwidth allocation
schemes for public clouds~\cite{Gui2014Soar, Guo2013, Niu2012, Popa2012, Shen2014NewBandwidth, Zheng2014Star}.
The emphasis in the cloud has been on having bandwidth reserved with service level guarantees for the cloud applications.

%% Gui2014Soar: SOAR: Strategy-proof auction mechanisms for distributed cloud bandwidth reservation
Gui et al.~\cite{Gui2014Soar} propose VCG-auction based mechanisms for reserving bandwidth 
at the multiple geo-distributed data centres of the cloud provider. 
The users submit bandwidth reservation requests separately for each of the data centres.
In case the users can accept partially fulfilled requests, 
i.e.~bandwidth reserved up to the maximum requested, 
a solution can be calculated using linear programming in polynomial time,
which achieves both optimal social welfare and strategy-proofness.
However, if partial reservations are not permissible, 
the allocation problem is NP-hard, so the above polynomial time algorithm
cannot be used to calculate optimal allocation. 
The authors propose a heuristics-based greedy algorithm that guarantees strategy-proofness,
though provides less than optimal social welfare.
%
%% Zheng2014Star: STAR: Strategy-Proof Double Auctions for Multi-Cloud, Multi-Tenant Bandwidth Reservation
Zheng et al.~\cite{Zheng2014Star} focus on a multi-cloud scenario where 
users need to reserve bandwidth from different cloud providers, 
because they have strict requirements on the amount of bandwidth for guaranteeing their quality of services. 
They model this open market of cloud providers as a double-sided auction, 
where providers also submit bids to the auctioneer besides the users,
and propose strategy-proof mechanisms based on McAfee double auction~\cite{Myerson1983}.

% Shen2014: New bandwidth sharing and pricing policies to achieve a win-win situation for cloud provider and tenants (INFOCOM)
Shen and Li~\cite{Shen2014NewBandwidth} propose a bandwidth pricing model, 
a network bandwidth sharing policy and flow arrangement policies, 
and use non-cooperative game theory analysis.
Their policies encourage cooperation among the tenants of the cloud infrastructure, 
who are incentivised to prefer uncongested links and constrain congestion.
%
% Guo2013: A cooperative game based allocation for sharing data center networks
Guo et al.~\cite{Guo2013} focus on the bandwidth available within the data centre infrastructure, 
on the links connecting the multiple VMs owned by the tenants.
They apply cooperative game-theoretic framework to design a distributed algorithm 
to achieve efficient and fair bandwidth allocation corresponding to the Nash bargaining solution.

%% Some other papers
%% Popa2012: FairCloud: sharing the network in cloud computing (SIGCOMM)
%% Niu2012: Pricing cloud bandwidth reservations under demand uncertainty

\subsection{Cloud Federations}
\label{sec__related_work_cloud_fed}

Aside from public clouds, another emerging model in cloud computing involves 
cloud providers trading of VM resources among themselves, 
referred to as federating their individual clouds.
There are various terms for this scenario in literature 
such as federated clouds, Intercloud, community clouds, cloud brokerage, etc. 
In this case, cloud providers agree to provision VMs for each other, 
for instance one provider can solicit additional resources from others to satisfy peaks in demands.
Such a federated or community cloud presents the challenge of a free market,
where participants have the incentive only to accept 
those resource exchanges that are profitable for them. % sentence paraphrased from Zhao2014
%% Zhao2014: Towards efficient and fair resource trading in community-based cloud computing
Zhao at al.~\cite{Zhao2014} develop a distributed market-oriented model for 
the resource negotiation and trading problem in such a community cloud.
They use this model to propose a multi-agent based approach 
that provides an efficient and fair resource allocation for a group of autonomous cloud providers.
They also consider resource trading under budget constraints, 
and based on a directed hypergraph model, present effective heuristic-based distributed protocols to achieve resource allocation within budget limits.

%% Social Cloud (Pairwise Matching, Virtual Credits and Currencies)
Social cloud computing~\cite{Caton2014, Punceva2013} similarly focuses on 
trading resources between cloud providers, 
who in this case are individual users of online social networks, like Facebook, Twitter, etc.
Punceva et al.~\cite{Punceva2013} propose a decentralised resource sharing model, and use virtual currency to incentivise cooperation 
without requiring a central reputation management system. 
They differentiate between intra-community and inter-community sharing, 
where a community consists of a group of \enquote{friends} on social networks, 
because the trust inherent within a tightly knit community aids in designing a more flexible virtual currency representation.
Caton et al.~\cite{Caton2014} focus more on improving how the providers are matched to the users in a social cloud.
They propose heuristics based matching algorithms for 
bidirectional preference-based socially-aware resource allocation, 
with the aim to optimise social welfare and allocation fairness.


%% Li2013: Virtual Machine Trading in a Federation of Clouds: Individual Profit and Social Welfare Maximization
Li et al.~\cite{Li2013Profit} study how individual cloud providers can maximise their profits through better resource trading and scheduling.
They apply a double auction-based mechanism,
which is strategy proof, individual rational, and ex-post budget balanced. 
Based on this auction mechanism, they propose an efficient and dynamic resource trading and scheduling algorithm, 
which carefully computes the true valuations of VMs in the auction, and aims to
optimally schedule stochastic job requests onto the VMs. 
%and judiciously turns on and off servers based on the current electricity prices. Through rigorous analysis, we show that each individual cloud, by carrying out our dynamic algorithm, can achieve a time-averaged profit arbitrarily close to the offline optimum.
%
%% Palmieri2013Distributed: A distributed scheduling framework based on selfish autonomous agents for federated cloud environments
Palmieri et al.~\cite{Palmieri2013Distributed} focus on scheduling resources within federated clouds, 
and present a fully distributed agent-based game-theoretic scheme 
for scheduling computing resources between providers in federated clouds.
Their scheme is based on independent, competing, and self-interested task execution agents, with the goal to achieve an optimum social welfare
criteria towards a Nash equilibrium solution,
using a slotted time model to provide advance reservation of resources in a fully distributed manner. 
%The agents' behaviour is also conditioned by marginal costs, to
%force some kind of implicit coordination between the (often conflicting) objectives of the various entities
%involved in the cloud. Due to its inherent parallel nature, such schema can provide a significantly better
%scalability in presence of a large number of tasks and resources involved into the scheduling system.

%% InterCloud

For the scenario of community network clouds, 
we consider members of community networks as cloud providers, 
and present efficient incentives based mechanisms for allocating cloud resources in \Cref{chap__incentives}.
