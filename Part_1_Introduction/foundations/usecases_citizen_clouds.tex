

\subsection{Citizen Community Clouds}
\label{sec__community_clouds_citizens}

Community clouds can also be built by the citizens in a bottom-up fashion through collaborative efforts~\cite{Khan2015Current}, 
by using resources contributed by individual users by either solely relying on user machines or using them to augment existing cloud infrastructures.
The idea of building cloud infrastructure using resources contributed by a community of users~\cite{Marinos2009}, 
follows on from earlier volunteer distributed computing platforms like
	SETI@Home~\cite{Anderson2002}, 
	BOINC~\cite{Anderson2004},
	Folding@home~\cite{Beberg2009}, 
	HTCondor~\cite{Thain2005}, 
	PlanetLab~\cite{Chun2003}, and
	Seattle~\cite{Cappos2009}, among many others,
and in general from the peer-to-peer systems~\cite{Shen2010} that focus on collaborative resource sharing.

There are a few such research prototypes for citizen community clouds that provide not the complete system but some of the components as a proof of concept.
%\subsection{Cloud@Home}
The Cloud@Home~\cite{Distefano2012} project aims to harvest resources from the community for meeting the peaks in demand, working with public, private and hybrid clouds to form cloud federations.
%The authors propose a rewards and credit system for ensuring quality of service.
The P2PCS~\cite{Babaoglu2012} project has built a prototype implementation of a decentralised peer-to-peer cloud system,
with basic support for creating and managing virtual machines using Java JRMI technology.
%It uses Java JRMI technology and builds an IaaS system that provides very basic support for creating and managing VMs. 
%It manages VM information in a decentralised manner using gossip protocols, however the system is not completely implemented and integrated.
The Clouds@home~\cite{Yi2011} project focuses on providing guaranteed performance and ensuring quality of service even when using volatile volunteered resources connected by Internet.
%Virtual Interacting Network CommunIty (Vinci) project~\cite{Baiardi2010} addresses the security issues involved in sharing private infrastructure with others through virtualization.
%\subsection{Personal Clouds}
Jang et al.~\cite{Jang2014} implement personal clouds that federate local, nearby and remote cloud resources to enhance the services available on mobile devices.
%
Social cloud computing~\cite{Chard2012, Caton2014} takes advantage of the trust relationships between members of the online social networks to motivate sharing of storage and computation resources,
by integrating with the programming interfaces (API) of the social network services which facilitates the establishment of mutual trust and resource sharing agreements.
%The requirements are that the system should closely integrate with the API provided by the online social networks providers.
%For instance, one of the social clouds projects~\cite{Chard2012} integrates Facebook API to provide a distributed storage service where one can store files on friends' machines.
%Users trade their excess capacity to earn virtual currency and credits that they can utilize later, and consumers submit feedback about the providers after each transaction which is used to maintain reputation of each user.
%%This scenario closely matches the community cloud problem and underlying issues and challenges are identical.
%Integration of social networks brings some complexity in implementing the system, but also makes issues like resource discovery and user management easier to tackle.
%Social clouds have also been deployed in CometCloud framework by federating resources from multiple cloud providers~\cite{Punceva2013}.
%Social compute cloud~\cite{Caton2014}, implemented as an extension of Seattle~\cite{Cappos2009} platform, 
%enables the sharing of infrastructure resources between friends connected through social networks,
%and explores bidirectional preference-based resource allocation.

% Some other European Projects: CloudSpaces, NanoDataCentres
