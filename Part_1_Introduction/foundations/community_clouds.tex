
\section{Community Network Clouds}
\label{sec__CN_Clouds}

The concept of community cloud computing has been introduced in its generic form~\cite{Mell2011} 
as a cloud deployment model in which a cloud infrastructure is built and provisioned for  exclusive use by a specific community of consumers with shared concerns and interests, 
owned and managed by the community or by a third party or a combination of both.
In this thesis, our focus is on the clouds built by the community, and for the community, 
relying on the resources available within community networks.
We first discuss the general idea of community clouds, before going into the detail of community network clouds.


% Sub-section on Community Clouds
\subimport*{\main/Part_1_Introduction/related_work/}{usecases_enterprise_clouds}

% Sub-section on Citizen Community Clouds
\subimport*{\main/Part_1_Introduction/related_work/}{usecases_citizen_clouds}

% Sub-section on Clouds in Community Networks
\subsection{Community Clouds in Community Networks}
\label{sec__CN_Clouds_Details}

% Sub-section on Community Networks
Community networks~\cite{Braem2013} are already based on the principle of sharing, though only of bandwidth, 
but the social aspects and community nature makes it easier to extend this sharing to other computing resources. 
The strong sense of community and technical knowledge of participants of such networks are some strong points 
which are conducive to building cloud applications tailored to local needs, built on infrastructure provided by the community members.


%% ------- Text from GECON'15
Community network clouds build on the success of community networks and aim to provide services and applications of local interest for the communities by applying the model of cloud computing.
Community network clouds fit nicely with the recent shift in exploring alternative approaches to large-scale data centres based public cloud computing, which include
	Inter-Cloud and federated clouds (where multiple public cloud providers work together),
	hybrid clouds (where enterprises combine their own cloud infrastructure with the public clouds),
	and edge clouds using nano data centres~\cite{Satyanarayanan2009} (where smaller clusters are deployed at the edges of the network to avoid latency and improve content-delivery).
These initiatives provide an excellent backdrop to explore the role of the community network clouds in enhancing the value proposition of the community networks, 
since an infrastructure of nano data centres~\cite{Satyanarayanan2009} to be deployed in a community network 
has to fit well with specific socio-economic and technical context of the community networks~\cite{Khan2014Sparks}.
For example, Guifi.net community network has deployed community edge cloud using their Debian-based Cloudy distribution~\cite{Baig2015Community}.
