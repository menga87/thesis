
\section{Summary}
\label{sec__related_work_conclusion}

The idea of community network clouds follows on from volunteer computing paradigm.
We looked at the general ideas of community clouds, both in commercial and non-commercial context,
and focused on the realisation of citizen community clouds within the community networks.
Community network clouds are as much a social construct as a technical one,
and so require careful design of resource regulation and allocation mechanisms,
in order to encourage participation by the members of the community networks.


\subsection*{Notes}

The background studies presented in this chapter (\cref{sec__CN_Clouds}) 
were accomplished in cooperation with my advisors Felix Freitag and Luís Rodrigues.
We are also thankful for the cooperation of Roger Baig and Roger Pueyo Centelles from Guifi.net, 
and Leandro Navarro from Universitat Politècnica de Catalunya (UPC),
who provided valuable insights into the operation of Guifi.net community network.
The studies were published as a book chapter 
	\enquote{\emph{Community Clouds}}~\cite{Khan2015CommunityClouds}, 
	in Encyclopedia of Cloud Computing (2016), published by Wiley \& IEEE (ISBN: 978-1-118-82197-8),
and as a short paper \enquote{\emph{Current Trends and Future Directions in Community Edge Clouds}}~\cite{Khan2015Current}, 
	in 4th International Conference on Cloud Networking (CloudNet 2015),
	Niagara Falls, Canada, October 2015.

The work on the proposal for distributed architecture presented in this chapter (\cref{sec__cloud_arch}) 
was accomplished in cooperation with my advisor Felix Freitag. 
We also collaborated with Mennan Selimi and Emmanouil Dimogerontakis, other PhD students at UPC and IST, 
who did provide relevant contributions for the evaluation of cloud services in community networks.
The proposal was published as a full paper
	\enquote{\emph{Towards Distributed Architecture for Collaborative Cloud Services in Community Networks}}~\cite{Khan2014Architecture},
	in 6th International Conference on Intelligent Networking and Collaborative Systems (INCoS 2014),
	Salerno, Italy, September 2014.
It was later integrated into an extended journal article
	\enquote{\emph{Cloud services in the Guifi.net community network}}~\cite{Selimi2015Cloud},
	in Computer Networks 93.P2 (Dec. 2015).
