
% incentives_motivation.tex

\section {Incentives Based Resource Regulation}
\label{sec__incentives_motivation}

Community network clouds, like other volunteer computing platforms, 
not only need to solve the technical challenges to be feasible but also 
address the social and economic context to engage their participants.
The existing social relationships among the users provide the foundation for building community cloud services.
Just as in social computing~\cite{Caton2014}, relationships from online social networks 
provide the backdrop to provision cloud services, 
in the case of community network clouds, we believe the existing
relationships among users of the community networks will make it possible 
for the model to be adopted.

The community networks provide the context and motivating scenarios 
for developing the proposed solutions that are discussed in Chapters \ref{chap__incentives} and \ref{chap__trusted_auction};
however, our proposals and findings are general enough for being applicable to other related fields.
In order to realise clouds in community networks, we can take the existing software and technology, 
and adapt and extend it to better fit with the characteristics and constraints of the community networks. 
For example, an initiative in this regard has been the Cloudy distribution~\cite{Cloudy}, 
which is based on Linux operating system and 
incorporates useful free and open-source services of interest to the members of Guifi.net~\cite{Selimi2015Cloud}. 
Of course, this does not imply that the community cloud model is restricted only to the members of the community networks.
In fact, with Cloudy distribution there has been an effort to engage and involve others who are not part of a community network, 
and let them connect using network tunnelling protocols~\cite{ClommunityTestbed}.
The emphasis is on building cloud services and applications of local interest 
using the resources contributed by the members of the community networks, 
but also on engaging the wider Internet community at the same time, 
which can also generate interest in the community networks themselves.

The resources available in the community network clouds can be divided into various categories.
Network resources, like upstream and downstream bandwidth at different links of the community network, 
available bandwidth at the Internet gateways, 
or bandwidth at the links connecting to the popular content servers, 
are limited and already of value to the members of the community networks.
Any community cloud solution has to work within the constraints of the available network capacity, 
and avoid negatively impacting the operations of the community network.
Community clouds at the basic level provide Infrastructure-as-a-service (IaaS), 
where resources like computation, memory and storage, are packaged as virtual machine (VM) instances.
In the collaborative model of community network clouds, users will be trading these resources as VMs among themselves. 
Other resources can include multimedia content, data storage, and other services, that are of interest to the users.

The underlying principles for regulating resources, in general, apply equally to all these different kinds of resources. 
In practice, the proposed solutions are often customised to better fit a particular problem scenario.
In our work, we have focused on two scenarios, which provide broad coverage of resource regulation problem in the community network clouds, 
and our prototypes addressing the two scenarios bring completeness to the solution for managing incentives in community network clouds. 

In \Cref{chap__incentives}, we focus on managing computing resources as VMs among the users.
We propose efficient regulation mechanisms in the context of a small community of users, 
perhaps part of a single zone of the community network, 
which already has trust and strong social relationships among its members.
Our mechanisms regulate usage of the shared resource of available VMs, 
and succeed in incentivising contribution from the users.
This is important in the community cloud model, since without active and continued contribution by the users,
the model cannot be sustainable and scalable.

In \Cref{chap__trusted_auction}, we broaden our scope to the overall community network 
where users participating in the community cloud may be from different zones, 
and have no prior relationships or established trust among them. 
We focus specifically on the bandwidth available at the Internet gateways in the community network for two main reasons.
Firstly, the available bandwidth to Internet is limited and 
already in high demand by the users of the community networks, 
and secondly, the community cloud services will also require access to Internet 
often with service-level guarantees for the bandwidth resource.
We propose a framework for distributed auctioneer that arbitrates users' access to the bandwidth 
available from the different getaway providers in a fair manner, 
even in the absence of trust among the users.
Even though in \Cref{chap__trusted_auction} we develop and evaluate the distributed auctioneer for bandwidth allocation, 
the proposed framework can be generalised and also directly applied to allocating VMs at a cloud service provider. 
We can also extend this framework for the scenario of trading VMs from \Cref{chap__incentives}, which we leave for the future work.
