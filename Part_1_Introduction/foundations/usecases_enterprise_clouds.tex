
\subsection{Commercial Community Clouds}
\label{sec__community_clouds_enterprise}

Community cloud is one of the many different deployment models for cloud computing~\cite{Murugesan2015Encyclopedia}. % ~\cite{Buyya2011CloudComputing}  
The most common and popular one, the public cloud, offers services of generic interest over the Internet, 
available to anybody who signs in with its credentials. 
On the other side, the private cloud model aims to provide cloud services to only a specific user group, 
such as a company, and the cloud infrastructure is isolated by firewalls avoiding public access. 
Finally, when a private cloud is combined with the public cloud, for instance some functionality of the cloud is provided by the public cloud and some remains in the private cloud, this is referred to as hybrid cloud.
The community cloud~\cite{Khan2015CommunityClouds}, bridges in different aspects the gap between 
the public cloud, the general purpose cloud available to all users,
and the private cloud, available to only one cloud user with user-specific services. 

% <copied> % this text Wiley Encyclopedia Chapter
Community clouds are implemented using different designs depending upon the requirements. 
One common approach is that a public cloud provider sets up separate infrastructure and develops services specifically for a community to provide a vertically integrated solution for that market. 
Similarly, a third party service provider can focus on a particular community and only specialise in building tailor-made solutions for that community.
Another option is that community members that already have expertise in cloud infrastructure come together to federate their private clouds and collectively provision cloud services for the community. 

Enterprises belonging to the same sector often use similar but independent cloud solutions, and comparing these solutions, it can be seen that these clouds are optimised in similar aspects which allow these enterprises to gain advantages for reaching common goals. 
Instead of these private clouds, building a community cloud for such enterprises shares the cost of the cloud solutions among them, and may also offer collaborations for mutual benefit even among competitors.
Such commercial community cloud solutions are a reality nowadays in several application areas and have been deployed in particular in the financial, governmental and health sector, besides many others, fulfilling the community-specific requirements, as evident from~\cite{NYSE2012, Optum2012}.

