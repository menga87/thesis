
Community clouds can have different interpretations depending upon the specific requirements and characteristics of the community, 
and how the infrastructures are deployed and services are provisioned.
We focus, in our thesis, 
on the collaborative model of provisioning community cloud 
services based on volunteer computing paradigm as laid out in~\cite{Marinos2009}.
In this chapter, we explore the background of community clouds,
and focus specifically on the problem of resource regulation.
%
The chapter is organised as follows.

\begin{itemize}
	\item We first look at the general idea of community clouds, 
	and how they are used in the commercial sector (\cref{sec__community_clouds_enterprise}).

	\item Next we focus on citizen community clouds built collaboratively 
	by the volunteers (\cref{sec__community_clouds_citizens}). 

	\item Then we study how the community networks provide an excellent context 
	for deploying community clouds (\cref{sec__CN_Clouds_Details}).
	
	\item We present the overall architecture for realising 
	a community cloud system (\cref{sec__cloud_arch}).
	
	\item We discuss different scenarios and contexts for applying 
	resource regulation in a community network cloud(\cref{sec__incentives_motivation}).
	
\end{itemize}

