
% Section: CONCLUSION
\section{Summary}
\label{sec__dist_auctioneer_conclusion}

Resource allocation is a fundamental problem in networked systems and the design of auction mechanisms that can provide properties such as
truthfulness, budget balance, and maximal social welfare have been extensively studied in the literature. 
These works assume a centralised trusted auctioneer that can faithfully execute the allocation algorithm. 
Unfortunately, many networked systems of today, such as \enquote{clouds of clouds}, edge clouds, and community networks, among others, lack a central trusted point of control (and, if it existed, it would be a bottleneck). 
In this chapter, we have addressed the theoretical and practical challenges that need to be overcome to bridge this gap. 
More precisely, we have proposed a novel distributed framework for devising Nash equilibria 
distributed simulations of the auctioneer that are resilient to asynchrony and coalitions. 
Furthermore, our framework allows for the parallelisation of the allocation algorithm, 
leveraging the distributed nature of the simulation, which is of paramount practical importance given that, 
in many allocation algorithms, achieving maximal social welfare is computationally intensive. 
We have devised implementations of the framework in a realistic testbed of one of the largest community networks deployed today, 
and have gathered experimental evidence that the overhead of the emulation 
is not significant, 
even in the cases the allocation algorithm cannot be parallelised, 
and brings substantial gains in the case parallelisation is possible. 
This shows that our approach can be used as a building block to implement resource allocation in decentralised networks.

\subsection*{Notes}
The results presented in this chapter were accomplished in cooperation 
with my co-advisor Luís Rodrigues and Xavier Vilaça, another PhD student at IST. 
Xavier did provide relevant contributions to the design of the game theoretical framework.

The study on the need for incentive-compatible pricing mechanisms (\cref{sec__truthful_pricing_introduction}) 
was presented in the paper 
	\enquote{\emph{Towards Incentive-Compatible Pricing for Bandwidth Reservation in Community Network Clouds}}~\cite{Khan2015Towards}, 
	in 12th International Conference on Economics of Grids, Clouds, Systems, and Services (GECON 2015), 
	Cluj-Napoca, Romania, September 2015.
The results for Distributed Auctioneer were published as a full paper 
	\enquote{\emph{A Distributed Auctioneer for Resource Allocation in Decentralized Systems}}~\cite{Khan2016Distributed}, 
	in 36th IEEE International Conference on Distributed Computing Systems (ICDCS 2016), Nara, Japan, June 2016.
