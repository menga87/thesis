
\section{System Model}
%\section{System Model for Distributed Auctioneer}
\label{sec__dist_auctioneer_model}
We define the family of resource allocation auctions, the requirements of 
a distributed simulation of the auctioneer, and the Game Theoretical model
used to analyse simulations.
These provide basis for our proposed framework for distributed auctioneer 
	(\Cref{sec__dist_auctioneer_design}).
	
\subsection{Resource Allocation Auctions}
Consider a family of auctions with $m$ \emph{providers},
$n$ \emph{users}, and an \emph{auctioneer}.
Providers sell multiple resources with a limited capacity,
in exchange for payments in some currency. Users are willing to pay
to the providers in exchange for the allocation of a minimum amount 
of each resource in the provider.
The auctioneer defines an allocation between users and providers
that is \emph{feasible} (i.e., that does not exceed 
the capacity of each resource in any provider)
and defines the payments to be made/received by the users/providers, respectively.
Both users and providers attribute a utility to each allocation,
which is a function of the value given to the allocation and the payments made/received.
More precisely, each user $i$ has a valuation $v_i$ 
specifying how much $i$ is willing to pay for the allocation
of a unit of each resource in each provider;
%\footnote{This models different preferences of users
%over different providers, which may be due 
%to differences in the hardware of providers, network latency, etc.}
$i$'s utility is the difference between the total value
attributed by $i$ to the allocation and the payments made by $i$.
On the other hand, the valuation $v_j$ of a provider $j$
specifies how much $j$ wants to be paid for allocating a unit of each resource;
$j$'s utility is the difference between the payments received by $j$
and the total value attributed by $j$ to the allocation.

Note that in the context of our model, 
\emph{providers} are the owners of the gateways, and have direct access to the Internet,
while \emph{bidders} have no direct access to the Internet and rely on these gateways.
The role of \emph{auctioneer} can be taken by one of the providers, or a chosen third party,
or in the case of our proposed distributed auctioneer, 
simulated by some providers in a distributed fashion.
We consider \emph{resource} to be the bandwidth external to the network available at the gateways.

We will analyse two types of auctions: \emph{standard} and \emph{double}
that differ only in who are the bidders (entities that submit bids).
In a standard auction, only the users are bidders.
Each user $i$ submits a bid $b_i$ to the auctioneer declaring $v_i$.
Then, the auctioneer executes an algorithm $\A$ that
returns a feasible allocation and the respective payments.
The algorithm $\A$ must satisfy three properties: (1) it must maximise, in expectation, 
the \emph{social welfare}, defined as the total value attributed by users to the allocation;
(2) it must achieve \emph{truthfulness in expectation}, i.e.
no user may increase its expected utility by lying in its bid;
and (3) it must be \emph{computationally efficient}.
In a double auction, 
the auctioneer collects bids from both the users and the providers.
The social welfare is now the
difference between the total value of the users and the total 
value of the providers. In addition to the above three
properties, $\A$ should also satisfy
\emph{budget balance}, which is the property that the total
value paid by users covers the total payments made to the providers.
Unfortunately, it was shown that no algorithm can simultaneously satisfy
truthfulness in expectation, maximal social welfare, and budget balance~\cite{Myerson1983}.
In practice, it is common to aim at a combination between
truthfulness in expectation and either one of the other two properties.

\subsection{Distributed Auctioneer Simulation}
In our setting, no single entity can be trusted with the role of the auctioneer,
since every entity may increase its utility by manipulating the execution of $\A$.
Specifically, a provider may devise a sub-optimal schedule
that provides him with a higher payment; similarly, a bidder
may manipulate the execution of $\A$ to decrease his payment.
We address this problem by simulating the role of the auctioneer through a distributed protocol.
The idea is to replicate the execution of $\A$ in multiple entities
and use cross-validation of the results of the redundant computations.
Providers are especially suited for this purpose, since they may be willing
to offer their resources for the execution of $\A$ in exchange for payments 
from the users.
Therefore, we focus on distributed protocols executed among sets
of providers that deviate from the protocol only if they gain by doing so.

Now, we specify requirements for a correct simulation of the auctioneer.
Normally, the auctioneer collects a vector $\vec{b}$
of bids and executes $\A$ with input $\vec{b}$.
In a simulation, each provider $j$ must collect a vector $\vec{b}^j$
of bids sent to $j$ and use it as input of a distributed protocol
that simulates $\A$. This requires bidders to submit a bid
to all providers. We consider that bidders may adopt arbitrary behaviours
such as submitting different bids to different providers or not submitting a bid.
Nevertheless, we assume that every provider $j$
eventually collects a vector $\vec{b}^j$ to be used as input in the simulation,
containing a bid for every bidder $i$, and if $i$ is correct,
then $j$ receives the bid of $i$ prior to the simulation.
In practice, bidders are expected to submit their bids by some deadline;
if a bidder fails to do so or sends an invalid bid,
then the provider may use the special value $\bot$ instead.
We want a simulation of the auctioneer to simulate $\A$ on some 
input $\vec{b}$ that contains at least the bids sent by correct bidders,
regardless of the bids of remaining bidders.

More precisely, let $\A(x,\vec{p} \mid \vec{b})$ be the probability
of algorithm $\A$ outputting an allocation $x$ and vector of payments $\vec{p}$,
when executed on input $\vec{b}$ by a trusted auctioneer.
We denote by $b_i^j$ the bid submitted by
bidder $i$ to provider $j$ in a simulation.
Let $\vec{b}^j$ be the vector of all bids sent to $j$.
If $i$ does not submit a valid bid to $j$, then we take $b_i^j$
to be a neutral bid (i.e., a bid that excludes $i$ from the auction).
In a simulation of the auctioneer, each provider $j$ inputs $\vec{b}^j$
and outputs a pair $(x,\vec{p})$ composed by an allocation $x$
and a vector of payments $\vec{p}$, or outputs a special value $\bot$
that signals the abortion of the simulation.
We say that the outcome is $(x,\vec{p})$ if all providers output this pair,
otherwise, the outcome is $\bot$.
This has the consequence that the mechanism is not resistant to faults or Byzantine behaviour,
since even a single missing or different value in the output can abort the whole allocation process,
and we leave it for the the future work (\Cref{sec__future_work}).
We assume that an external mechanism guarantees that
(1) when the outcome is $\bot$, the auction is aborted,
and (2) when the outcome is $(x,\vec{p})$, the allocation $x$
is enforced and all entities perform or receive their respective payments.
We can now provide a precise definition of correct simulation.

\begin{definition}
A simulation is said to be correct if and only if,
for all vectors $(\vec{b}^j)_j$, the outcome is $(x,\vec{p})$ with
probability $\A(x,\vec{p} \mid \vec{b})$, where $\vec{b}$
only contains valid bids and, for all bidders $i$ such that 
$b_i^j = b_i'$ for every provider $j$, we have $b_i = b_i'$.
\end{definition}

\subsection{Game Theoretical Model}
We consider the model of extensive form games played in 
asynchronous systems proposed in~\cite{Abraham2013}.
There are $m > 1$ players corresponding to the providers 
of the resource allocation auction.
Providers may form coalitions of size at most $k$. 
We assume that each provider has a unique identifier,
known to every other provider. Time is divided into turns.
In each turn, some provider $j$ is chosen to move:
$j$ first receives messages sent to it in the previous round,
performs some computation, and sends messages.
A schedule specifies which provider moves at each turn
and which messages it receives.
We assume that communication channels are reliable,
so every message sent is eventually delivered.
We focus on schedules that are \emph{fair} in the sense
that every provider $j$ is scheduled to move infinitely
often, such that, for all turns $t$, there exists a turn $t'>t$
when $j$ is scheduled to move. This is necessary to ensure progress.

Now, we want to define a notion of equilibrium for this setting.
For this, we need an exact definition of protocol and utility.
A protocol specifies, for each schedule, a probability distribution over
the computation performed by each provider $j$ and the messages sent at each turn where $j$ moves,
as a function of the history of messages sent and received in previous turns.
In addition, since we analyse protocols as modules with input and output values,
the protocols also specify the values used as input and output by each provider.
The utility of providers is a function of the outcome of the simulation:
if the outcome is $\bot$, then the utility is $0$, else the
utility is the difference between the payments received and the value of the allocation.
Given this, the utility of a user $i$ is also $0$ if the outcome is $\bot$,
or is the difference between the value of the allocation and the payments made.
The expected utility conditioned on the schedule is
computed according to the probability distribution over outcomes induced by the protocol.

For the definition of equilibrium, we consider the notion of $k$-resilient 
(ex post) equilibrium introduced in~\cite{Abraham2013}.
This notion is a refinement of Nash equilibrium that incorporates
collusion and asynchrony. Namely, collusion is modelled as sets $K$
of at most $k$ providers that coordinate on any joint protocol;
we require that no provider in $K$ can increase its expected utility
if providers in $K$ deviate from the specified protocol,
given that other providers do not deviate.
Asynchrony is modelled in an ex post way by assuming that providers 
are informed about the schedule when computing the expected utility.
This is the strongest requirement for this setting.

\begin{definition}
A protocol $P$ is a $k$-resilient (ex post) equilibrium if and only if
for all fair schedules and coalitions $K$ such that $|K| \le k$,
there is no provider in $K$ that increases its expected utility
when providers in $K$ follow a joint protocol $P' \ne P$,
given that providers not in $K$ follow $P$.
\end{definition}

An important aspect of this definition is that a $k$-resilient
equilibrium protocol $P$ satisfies the property that no provider increases
its expected utility by lying about its input, hence $P$
achieves \emph{truthfulness} regarding the inputs of the providers to a simulation.
Specifically, at every invocation, we say that provider $j$ has input $v$
if, by following $P$, $j$ is expected to input $v$, so truthfulness implies that $j$ 
does not input $v' \ne v$.
By the truthfulness of $\A$, every bidder $i$ maximises its expected utility
by sending $b_i = v_i$ to all providers, regardless of other bids.
This implies that to fulfil our goals it suffices to devise
protocols that are $k$-resilient and correctly simulate the auctioneer.

