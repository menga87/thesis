
% Section: SYSTEM MODEL
\subsection{System Model}
\label{sec:model}

We consider a bandwidth provider $\mathcal{P}$ in the community network and a set of N users $\{1, 2, \ldots N\}$. % $\mathbb{P}$
The provider operates a gateway to the Internet, which allows access to resources outside the community network for the users.
The users are connected to the provider's gateway through the wireless and fibre links in the community network~\cite{Baig2015Guifi}, 
and the applications in community network cloud access their external servers and Internet through this gateway.
Figure~\ref{fig__external_bandwith_mesh_network} shows the users in the community network connected to the provider through multiple such paths, where only few of these users are the clients of the provider for reserving bandwidth.

The provider processes the requests in a queue at the gateway, where time in the queue is divided into an infinite sequence of slots starting from $1$, 
where all the available bandwidth is allocated to exactly one user in each slot.
The provider allocates the slots in batch after receiving all the requests from $N$ users and assigns the next $N$ slots, one to each user.
We choose this model for simplicity, and we divide users into different priority classes based on their slots in the queue, as discussed below.
Our findings are applicable to other models, for instance where all the available bandwidth is shared between the users at the same time.
In this case, we can have priority classes based on the quality of different links, for example.
Considering the provider has two links to the Internet, high priority requests are assigned to the better link with low latency, less packet loss, etc., 
while the low priority requests are assigned the other link.
The discussion and findings presented below can equally apply to this alternate model as well. 

We divide the users into two priority classes, $h \in \{0, 1\}$,
some have lower priority requests, $h_0$,
and some have higher priority requests, $h_1$.
Here in this model, the main consideration for higher priority requests is that they are more sensitive to the waiting time,
and prefer to reserve earlier slots in the queue.
Provider $\mathcal{P}$ aims for an optimal schedule when allocating the slots to the users, 
so as to maximise its revenue
and the overall utility for all the users.
We provide formal details below.

\paragraph{Schedule} A schedule $\phi$ maps each time slot $t$ to a user $i$.

\paragraph{Value} For any schedule $\phi$ and user $i$, let $t$ be the
slot assigned to user $i$, then 
$v_i(h, t)$ is the valuation given by $i$
for being allocated time slot $t$,
where $h \in \{0, 1\}$ is the priority class of the user.
$v_i$ is communicated by each $i$ to $\mathcal{P}$ beforehand.

\paragraph{Utility} For any schedule $\phi$ which assigns user $i$ a slot $t$, the utility $u_i(\phi, t)$ for user $i$ is difference between the value $v_i$ and the payment $p_i(\phi, t)$ made by user $i$ to $\mathcal{P}$.

\begin{equation}
	u_i(\phi, t) = v_i(h, t) - p_i(h, t)
\end{equation}

\paragraph{Restriction} Any slot $t$ can be assigned to at most one user.

\paragraph{Optimization} Find $\phi$ that maximises the social welfare, which is the sum of utilities $u_i$ of all users, while fulfilling the restrictions.

\begin{equation}
	\text{maximise} \ \ \textrm{\itshape welfare}\, (\phi) = \sum_{i \in N} u_i(\phi,t) % = \sum_{i \in N} \left(v_i(h, t) - p_i(\phi, t)\right)
\end{equation}

\paragraph{Scheduler} Function $S$ that maps $\vec{u} = (u_i)_{i \in N}$ to optimal $\phi$.

\paragraph{Goal} A user $i$ when submitting the request to $\mathcal{P}$, declares the priority class $h_i$ and value $v_i$, and also the bid amount $b_i$ where applicable.
When the user behaves truthfully the reported value $v^*_i$ is the same as her inherent value $v_i$.
We want to ensure that it is in the interest of every $i$ to declare her true value of $v_i$,
regardless of the declared values $v_j$ for any $j \neq i$. 
Such a mechanism is said to be truthful in dominant strategy,
where users have no incentive to misreport their values~\cite{Nisan2001}.


\subsection{Pricing Mechanisms}
\label{sec:model-pricing}
Given the above model, the prices are calculated for the bandwidth usage according to different mechanisms~\cite{Maille2014}.

\paragraph{Fixed Pricing}
In the case of fixed usage-based pricing, all the users pay the identical price $c_0$ for each unit of bandwidth consumed, which is constant irrespective of the priority class.

\paragraph{Priority Pricing}
In the case of priority pricing, users pay according to the priority class $h$. 
Since in our model, there are only two priority classes $h_0$ and $h_1$, provider $\mathcal{P}$ charges two different prices $c_{h_0}$ and $c_{h_1}$ per unit of bandwidth, respectively.

\paragraph{First-Price Auction}
In the case of sealed first-price auction, users make different bids $b_i$ depending on their priority class $h$, with high priority requests quoting higher bid amounts in general. 
Each winning user pays their bid amount.

\begin{equation}
	p_i(\phi, t) = b_i
\end{equation}


\paragraph{Generalised Second Price (GSP) Auction}
In a generalised second price (GSP) auction, users make different bids $b_i$ but in this case the winning user pays the amount corresponding to the next highest bidder~\cite{Jana2011}. 
So the user with the highest bid pays the amount of the second highest bidder, and the second highest bidder pays the amount of the third highest bidder, and so on. 


\paragraph{Vickrey-Clarke-Groves (VCG) Auction}
VCG is a second-price sealed-bid auction based mechanism, which ensures truthfulness and maximum social welfare~\cite{Nisan2001},
if the provider $\mathcal{P}$ can calculate optimal schedule $\phi$ in polynomial time.
Each user $i$ provides a bid $b_i$ to $\mathcal{P}$, and given a schedule $\phi$, each user $i$ pays the price $p_i(\phi, t)$ according to:

\begin{equation}
	p_i(\phi, t) = \sum_{\substack{
						j!= i \\ 
						j \in N}} 
					\left( v_j(h,\phi') - b_j \right) - 
					\sum_{\substack{
						j!= i \\ 
						j \in N}} 
					\left( v_j(h,\phi) - b_j \right)
\end{equation}

where $\phi$ and $\phi'$ are the schedules that maximise $\displaystyle\sum_{i \in N} u_i$ while including and excluding the bid $b_i$ by user $i$ from the allocation respectively.

\subsection{Scheduling Algorithm}

We consider a simple scheduling algorithm which applies a greedy approach for mapping users' requests to the available slots.
Algorithm~\ref{alg:scheduling} shows the scheduling algorithm, where $\mathcal{P}$ assigns the slots to the users in non-increasing order of their reported bids (and corresponding $h_i$ and $v_i$) for the bandwidth resource.
The prices calculated are dependent on the pricing mechanism, the priority class $h$ of the requests, and the assigned slot $t$ in the schedule $\phi$.
The runtime of the algorithm is $O(N\, log\, N)$ for $N$ users for the different pricing mechanisms.
VCG mechanism, however, requires computing $N$ schedules for calculating payments for the $N$ winning bids, so the running time in the case of VCG is $O(N^2\, log\, N)$. % \mathcal{O}

The greedy approach, in general, does not always provide an optimal allocation, which is a pre-requisite for VCG mechanism.
However, in the case of the model given above and considering the step function we are going to use for $v_i(h, t)$ from Figure~\ref{fig:value_fn}, the greedy approach from Algorithm~\ref{alg:scheduling} always returns an optimal allocation. 
This can be proven through induction, and can be explained intuitively as follows.
Selecting the requests with higher bids first (corresponding to higher $h_i$ and $v_i$) will always give the maximum social welfare,
since the value function in Figure~\ref{fig:value_fn} is non-increasing with time and choosing a bid with lower amount causes a loss in social welfare which cannot be recovered as the time progresses. 

%%NOTE: avoid two arrows on top $\protect\vec{t}$

\begin{algorithm}[tbp]
	\caption{Scheduling algorithm for $\phi$, allocating $\protect\vec{t}$ slots to $N$ users}
	\label{alg:scheduling}
	\small
	
\begin{algorithmic}[1]
\Require List of users  $\vec{n}$, bids  $\vec{b}$, for total $N$ users
\Ensure List of assigned slots $\vec{t}$, and payments $\vec{p}$ 

\State Sort  $\vec{n}$ users in non-increasing order on their bids  $\vec{b}$

\For {$i = 1, \ldots N $}
	\State $ \vec{t}[i] \gets \vec{n}[i]$
	\Comment Assign slots
\EndFor

\For {$i = 1, \ldots N $}
	\State $ \vec{p}[i] \gets \text{\emph{payment}}\,(\vec{b}[i],
	\vec{t}[i])$
	\Comment Calculate payments
\EndFor

\end{algorithmic}

\end{algorithm}
