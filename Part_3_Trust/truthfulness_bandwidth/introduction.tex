
% Section: INTRODUCTION
%\section{Incentive-Compatible Pricing for Bandwidth Reservation}
%\section{Ensuring Truthfulness in Resource Allocation}
\section{Motivations}
\label{sec__truthful_pricing_introduction}

We have seen in \Cref{chap__incentives} how incentives based resource regulation is important to
encourage contribution and thus ensure a sustainable community cloud ecosystem.
However, the mechanisms in \Cref{chap__incentives} assume the users will always follow the prescribed polices,
and just like in community networks, the social context and the ties between the community
would be sufficient to correct any erroneous behaviour.
But these assumptions do not always hold true, especially when the community cloud would grow  to a large user base.
With weakening of any direct social interaction, the system may be open to abuse by the selfish or malicious users.
This can negatively impact the viability of community cloud model.

%\minorin{}{Write more on motivations and link to the previous chapter.} %TODO

In order to understand better the implications of untruthful behaviour, 
we study the effects of untruthfulness on the utility obtained by the users and the social welfare of the overall system~\cite{Khan2015Towards}.
We first present a model that differentiates between cloud applications with different priority classes.
Next, we use this model to evaluate the impact of untruthfulness on the social welfare
through simulation experiments, 
for the different pricing mechanisms proposed in the literature~\cite{Maille2014}.
