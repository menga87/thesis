

\section{Ramifications and Collaborations}
\label{sec__collaborations}

The research in this thesis was extended in other contexts through collaborations.
We discuss the details below, and comment on their relevance to the work presented in this thesis.

\subsection{Community Clouds}
\label{sec__community_clouds_background}

We study the general idea of community clouds 
in the context of other prevailing cloud models, in particular for the industry~\cite{Khan2015CommunityClouds}.
We investigate how this extends to community clouds built collaboratively,
realising ideas from edge computing and volunteer computing~\cite{Khan2015Current}.
This provides the broad context for community network clouds, 
and indicates their potential use cases.

\subsection{Social and Economic Mechanisms}
\label{sec__economic_mechanisms}

We explore the social and economic mechanisms that can help in adoption and growth of community network clouds~\cite{Khan2014Macroeconomic, Khan2014Sparks}. 
We take into account the social and technical context of community networks,
and present a cost-value proposition describing the conditions under which community network clouds could emerge. 
We propose a set of technical, social and economic mechanisms that, 
if placed in community networks, 
can help accelerate the uptake and ensure the sustainability of community network clouds.
These mechanisms highlight the importance of managing incentives in community network clouds.

\subsection{Scalability of Community Cloud Architectures}
\label{sec__scalability_background}

We investigate the scalability of community network cloud model~\cite{Khan2013Clouds}
in order to analyse different aspects of its performance 
in comparison to public clouds through simulation experiments.
In a community network cloud the computing resources are heterogeneous and less powerful, 
but are geographically distributed and are located closer to the users.
Our results suggest that the performance of the community clouds 
depends on the conditions of the community networks, 
but has potential for improvement with network-aware cloud services. 

\subsection{Supporting Service Selection} 
\label{sec__service_selection}

Community network clouds need support mechanisms that provide assistance 
in cloud service selection while taking into account 
different aspects pertaining to associated risks in community clouds, 
quality concerns of the users and cost limitations specifically in multi-clouds ecosystems. 
We propose a risk-cost-quality based decision support system~\cite{Khan2015DSS} 
to assist the community cloud users to select the most appropriate cloud services meeting their needs. 
The proposed framework not only increases the ease of adoption of community clouds 
by providing assistance to users in cloud service selection, 
but also provides insights into the improvement of community clouds based on user behaviour.


\subsection{Cloud Services in Guifi.net}
\label{sec__cloud_services_background}

We materialise the proposed framework for community network cloud~\cite{Jimenez2013, Freitag2014Energy, Baig2015Community, Selimi2015Cloud, Khan2015Enabling}
in the implementation of the \emph{Cloudy} distribution~\cite{Cloudy}. 
This distribution can be used to integrate useful services and applications 
that provide value to end-users of the community network cloud.
We conduct real deployments of these clouds in the Guifi.net community network 
and evaluate cloud-based applications such as service discovery and distributed storage. 
This deployment experience supports the feasibility of community clouds, 
and our measurements demonstrate the performance of services and applications running in these community clouds. 
Our results encourage the development and operation of collaborative cloud-based services 
using the resources of a community network, 
and we anticipate that such services can effectively complement commercial offers.




%Inspired by: http://www.gsd.inesc-id.pt/~ler/reports/joaoleitaophd.pdf
%See section on "Ramifications and Collaborations" in Chapter 7.1.
%We could list other work and publications.