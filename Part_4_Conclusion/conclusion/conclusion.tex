

Cloud computing with its success in providing virtualised resources on demand has transformed the technology landscape, revolutionising how Internet applications are developed and delivered to the users.
Peer-to-peer and edge computing models have been explored in the past decade, but other than a few success stories they never made it big in the mainstream.
Perhaps now is the right opportunity to take full advantage of the virtualization model of the cloud computing to design the killer applications for the community cloud.
On the technical side, this allows for sophisticated applications and services that were not possible with the simple process-level isolation approaches of earlier efforts like BOINC or Seattle.
The challenge is to provide developer tools and middleware services that streamline the process of programming and deploying the community cloud applications.
At the same time, killer applications are needed that by satisfying users' critical needs and problem scenarios succeed in engaging the community for the long-term.

In this thesis, we looked at the field of community clouds in general, 
and clouds in community networks in particular,
to develop economic regulation mechanisms in tune with 
the specific social, economic and technical context of the community networks
in order to help us in developing and sustaining a community cloud ecosystem.
We looked at how these mechanisms fit in an overall framework of a community cloud system,
which supports components for 
	incentivising contribution, 
	regulating access to resources, 
	and ensuring trust in the system.
We developed incentive-based resource regulations mechanisms for a well-knit community of trusted users,
and showed how they are crucial for successful operation of a community cloud.
We also noticed how lack of trust in a community cloud deployed on a large scale affects negatively the utility and sustainability of the system.
To address this issue, we developed a distributed auctioneer component, 
which is a virtual trusted entity integrated in the overall architecture of the community cloud.
This component allows to efficiently and optimally allocate resources in a community of untrusted users,
with negligible communication and computation overhead, 
and it can also leverage parallelised implementation to offer scalability.
