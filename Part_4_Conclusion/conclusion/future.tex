
% FUTURE WORK
\section{Future Work}
\label{sec__future_work}

%%-------- FROM AINA 2014 -----
Carrying onwards from the experience and results with the prototype implementations, 
a working service needs to be developed further, 
that provides the feedback loop between the users' contribution and experience, 
and will be inevitable for adoption, sustainability, maintenance and growth of cloud infrastructures in community networks. 
Larger scale deployments are required with extended implementation of the different components of the community cloud framework.
This should be complemented by additional services and applications deployed in the cloud infrastructure, 
which will provide enhanced value and utility to the members of community networks for their contribution towards the community cloud. 

%%-------- FROM GECON 2015 -----
With respect to bandwidth allocation mechanism,
there are others challenges, for instance, multiple users may be connected to the provider 
using the same path in the community network, 
and reserving bandwidth for such users in the same time interval 
may cause congestion across some of the links, 
which an intelligent allocation algorithm should try to avoid.
Moreover, any bandwidth reservation scheme should not negatively impact the normal operation 
of the community network, 
so allocation mechanism needs to be adaptive to the network congestion and bandwidth usage in the community network.

%%-------- RELATED TO INFOCOM 2015 -----
For distributed auctioneer, we have considered only rational users, and we plan to extend our framework to the Byzantine users.
In the current model, we assumed all providers to be fully inter-connected. 
But in the case of federated community clouds, it is possible that
providers in different local clouds may not have very good connections between them, 
and so the current approach may result in slow down.
In such cases, we plan to explore how to adapt our distributed auctioneer to different network topologies.
%In this case, it could be better to have a tree like structure with providers calculating values in sub-trees near the leaves, 
%values being propagated up the tree, until we have a final outcome at the root. 
